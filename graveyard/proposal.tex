%%%%%%%%%% METHODS %%%%%%%%%%%
\section{Methods}
\label{sec:methods}

% in the vicinty of
% clarify that  flow is highly 3D with wind veering / backing, strong zonal wind higher up stratosphere  favors propagation 

% simulations are simplified for now only considering coriolis effect based on obstacle deflection??

% Maybe include: \\
% - no baroclinic instability simulation -> no baroclinic jet stream \\
% - only barotropic jet

The goal of this thesis is an improved understanding of the processes that lead to observed gravity waves above TDs. A complete three dimensional simulation of the troposphere and stratosphere including all relevant features, more precisely a Rossby wave train and jet stream at the tropopause and a PNJ higher up in the stratosphere to the south, requires challenging initial conditions to obtain suited fields of wind, temperature and pressure (compare Section \ref{sec:barocInstability}). Though such a simulation has to be the final step to prove or debunk the described mechanisms, preliminary investigations of simplified simulations likely provide first insights and might significantly help to setup and interpret more complex runs. Therefore, the proposed thesis comprises a selection of simulations with reduced complexity to address parts of the problem step by step.

The fundamental simplification for all preliminary simulations is the focus on dynamics in the stratosphere by introducing the tropopause as the lower boundary of the simulation's domain. It is clear that the tropopause is not an impermeable surface, but observations from radiosondes (\cite{birner_how_2002} and \cite{birner_fine-scale_2006}) and GPS data (\cite{randel_extratropical_2007}) revealed a significantly higher thermal stability of the extratropical tropopause inversion layer compared to model and reanalysis data. This higher stratification further inhibits troposphere-stratosphere exchange at the tropopause and justifies the approach of using the tropopause as the lower boundary for first simulations. The planned simulations and corresponding simplifications are described in the following subsections and summarized in table \ref{tab:simRuns}.

\begin{table*}[]
\caption{EULAG simulation runs}
\begin{tabular}{@{}lll@{}}
\toprule[1pt]
                    & Simulation                                             & Description                                                                                                                                                          \\ \midrule[1pt]
\multirow{19}{*}{2D} & 001: Fundamental flow regimes over orography                & \begin{tabular}[c]{@{}l@{}}- Non-hydrostatic wave regime\\ - Hydrostatic wave regime\\ - Inertia-gravity wave regime \\ \end{tabular}                                                                      \\ \arrayrulecolor{black!30}\cmidrule[1pt]{2-3}
                    & 002: Mtn / TD shape comparison                & \begin{tabular}[c]{@{}l@{}}- Witch of Agnesi\\ - $(1+cos(\phi))$ shape \\ - $(1+cos(\phi))^4$ shape \end{tabular}                                                                      \\ \arrayrulecolor{black!30}\cmidrule[1pt]{2-3}
                    & 003: Transient lower boundary test                              & \begin{tabular}[c]{@{}l@{}}- Mtn rises or moves in $x$-direction \\ - No background wind  \\ - Test smooth start up / end of motion \end{tabular}        \\ \cmidrule[1pt]{2-3}
                    & 004: Transient TD like Prusa et al. (2003)               & \begin{tabular}[c]{@{}l@{}}- Oscillating TD in zonal direction \\ - Constant stratospheric background wind \\ - Constant stratospheric stability \end{tabular}         \\ \cmidrule[1pt]{2-3}
                    & 005: Propagating TD with vertical shear  & \begin{tabular}[c]{@{}l@{}} - PNJ at $\approx 40$km  \\ - TD moves with phase velocity of Rossby wave \\ - Design idealized wind profile \\ - Use wind profile from ECMWF \\ - \textbf{Sensitivity analysis} wrt. shape of depression \\ ($h_0$, $a$ and asymmetry)    \end{tabular}      \\ \cmidrule[1pt]{2-3}
                    & 006: Propagating TD with meridional wind & \begin{tabular}[c]{@{}l@{}} - Run 2D005 with Coriolis force \end{tabular}        \\ \arrayrulecolor{black}\cmidrule[1pt]{1-3}
\multirow{11}{*}{3D}                 & 007: Propagating TD with vertical shear                & \begin{tabular}[c]{@{}l@{}} - Run 2D005/006 in 3D  \\ - TD oriented N-S \\ - Compare elongated and local depression \\ (elliptic shape) \end{tabular}   \\ \arrayrulecolor{black!30}\cmidrule[1pt]{2-3}
                    & 008: Tilted TD with vertical shear                     & \begin{tabular}[c]{@{}l@{}} - Run 3D007 with tilted TD (oriented NW-SE)  \end{tabular}       \\ \cmidrule[1pt]{2-3}
                    & 009: Barocl. (or barotropic) PNJ above TD  &  \begin{tabular}[c]{@{}l@{}} - Include horizontal shear \\ - 2D Gaussian distribution (for barocl. jet) \\ - $\theta_{env}$ from thermal wind relation (for barocl. jet) \\ - PNJ directly above tropopause depression  \end{tabular}    \\ \cmidrule[1pt]{2-3}
                    & 010: Barocl. (or barotropic) PNJ shifted south   & \begin{tabular}[c]{@{}l@{}} - Simulation 3D009 with PNJ shifted south \end{tabular}      \\ \cmidrule[1pt](l){2-3} 
                    & 011: Full simulation including troposphere                  & \begin{tabular}[c]{@{}l@{}}- Initialisation based on Bush et al. (1994)\\ - Extension of barocl. instability with PNJ\end{tabular}     \\
                    \arrayrulecolor{black}\bottomrule[1pt]
\end{tabular}
\label{tab:simRuns}
\end{table*}

\subsection{Model setup and verification}
\label{sec:modelVerification}

% 8). In effect, the characteristic Froude number Fr = N hm /U = 0.1 < 1. Therefore, according to [36], we can expect that the steady nonlinear solution of the numerical simulations is close to the linear one.

Simulations 2D001-2D004 of table \ref{tab:simRuns} ensure that the model is setup correctly and provides physically reasonable outputs. In that sense, it is the goal to reproduce some fundamental analytic results based on linear theory with nonlinear EULAG simulations. \textcite{queney_problem_1948} was the first to summarize different flow regimes over mountains based on the method of small adiabatic perturbations. His results can be compared to simulations from a visual perspective by looking at the perturbations and fluxes in the atmosphere and from a mathematical one by comparing the wave drag
%
\begin{equation}
    \mathcal{F} = \int_{}^{} p^{'} \frac{\partial h}{\partial x} dx
    \label{equ:waveDrag}
\end{equation}
%
to its analytic solution in a linear framework (\cite{gill_atmosphere-ocean_1982}). Assuming an impermeable tropopause at the lower boundary of the simulations results in a TD that acts like a flipped mountain (valley) on the stratosphere, so a comparison of our model to Queney's solutions for different mountain wave regimes (non-hydrostatic, hydrostatic, inertia-gravity) provides a reasonable baseline and validation of the model for the planned simulations.

Depending on the application or numerical problem at hand, the function, that describes the surface boundary (mountain shape) can have useful or disturbing properties. The Witch of Agnesi

\begin{equation}
    h_{Agnesi}(x) = \frac{h_0}{(\frac{x}{a})^2+1}
    \label{equ:witchOfAgnesi}
\end{equation}

used by \textcite{queney_problem_1948} is a convenient equation of the terrain height for analytical analysis, but it is non-zero for the whole domain of discrete simulations. In the case of a transient lower boundary this effect leads to inconsistent boundary conditions at the horizontal borders, making this a less appropriate shape for a propagating TD. A suitable alternative is a $1+cos(\pi \phi)$ function. It drops to 0 for $|\phi| = 1$. With 
%
\begin{equation}
    h_{cos}(\phi) = \frac{h_0}{16} (1+cos(\pi \phi))^4
    \label{equ:cosMtn}
\end{equation}
%
and $ \phi = \frac{x}{4a}$ \textcite{epifanio_three-dimensional_2001} used a variation of this function, which is more comparable to the Witch of Agnesi (\ref{equ:witchOfAgnesi}). Setting $h_{cos}(\phi)=0$ for all grid points $|x| > 4a$ results in a continuous and differential lower boundary that does not interfere with horizontal boundary conditions as long as the mountain or depression is not too close to the edges. Furthermore, all contributions to the surface pressure drag are confined to a small neighbourhood around the mountain peak. \textcite{metz_are_2021} state that the wave drag is a factor 1.3 higher compared to the Witch of Agnesi, but further comparisons of the terrain functions and their impact on the overlying flow are carried out (simulation 2D002).

In simulations 2D003 it is the goal to observe the propagation of waves for a rising and for a moving lower boundary in an atmosphere at rest. With no background wind and a constant static stability the simulation is expected to mimic the perturbations in a water tank, when for example pulling an obstacle at the bottom through the quiescent fluid. Furthermore, in this way the linearized governing equations simplify even further and it might be possible to identify patterns in the waves' group ($c_g$) and phase velocities, which are characteristic for the scenario without a background flow. For example the parallel orientation of $c_g$ to the wave's phase lines and the corresponding direction of the energy propagation (\cite{lin_mesoscale_2007}).

For a last physical validation it is appropriate to start two-dimensional simulations of a transient TD by reproducing the simulations of a zonally oscillating TD conducted by \textcite{prusa_all-scale_2003} (simulation 2D004 in table \ref{tab:simRuns}). They showed that inertia-gravity waves are constantly observable above a TD ($h_0$=\SI{500}{\meter}, $a$=\SI{200}{\kilo\meter}) for a constant stratospheric background wind of \SI{10}{\meter\per\second}, a constant buoyancy frequency $N$ of \SI{0.02}{\second^{-1}} and an isothermal stratosphere. However, the waves above the TD are expected to change continuously, because the relative speed of the depression with respect to the constant background wind changes, too. In particular, the varying tilt of the phase lines should be observable. 

\subsection{2D simulations of a propagating tropopause depression}
\label{sec:2D}

The next step in the two-dimensional space is the introduction of a realistic stratospheric wind profile (simulation 2D005 in table \ref{tab:simRuns}). Two approaches are possible. The first one incorporates the design of idealized profiles with increasing complexity. At first, a PNJ could be represented by a Gaussian distribution peaking at a level between 40-\SI{50}{\kilo\meter}. In a next step, it could be of interest to add a second jet at the tropopause level resulting in a negative wind shear in the lower stratosphere before the wind increases again higher up. For this purpose, a higher order function approximation might be more suited than a superposition of two Gaussian distributions to avoid sharp changes in the vertical shear. Environmental profiles of $\theta$, $T$ or $p$ still refer to a constant stability and an isothermal atmosphere. The second approach relies on realistic ECMWF wind profiles, which are already available from the observational analysis of RF25 of the DEEPWAVE campaign (\cite{dornbrack_stratospheric_2021}). These profiles represent meridional means of a relevant 5$\degree$ wide latitude band and naturally include complex shear scenarios due to the PNJ and the tropopause jet stream. 

For these simulations (2D005) it is planned to reduce the movement of the TD to a constant propagation from left to right through the simulation domain. In this way, the effect of vertical wind profiles on the excitation and propagation of waves above a transient TD is not disturbed by an unrealistic movement of the depression. At the upper and lateral boundaries, the vertical and horizontal radiation of wave energy is treated by relaxation terms as further described in Section \ref{sec:EULAG}. The parameters of these damping layers are tuned to reduce boundary effects like wave reflection to a minimum.

At this point, it is also planned to conduct a sensitivity analysis with respect to the depression's shape. Effects of its depth, width and possibly asymmetry shall be investigated. The settings used by \textcite{prusa_all-scale_2003} serve as a good starting point, but observations (\cite{bush_tropopause_1994} and \cite{keyser_review_1986}) and reanalysis data (\cite{dornbrack_stratospheric_2021} and \cite{skerlak_tropopause_2015}) suggest quite a range of realistic parameters. This sensitivity analysis can provide first hints on expected wave regimes and help to identify optimal settings for further simulations.

Simulation 2D006 is closely related to 2D005 with the difference of activating the Coriolis force and allowing meridional winds. All other settings will be left unchanged with respect to a reference simulation of 2D005 to directly observe differences in the wave excitation and propagation.  

% Cyclic boundary conditions?? interaction with boundaries again! 

% cite RF25 paper for profiles

% max. wind speed between 40-80 m/s (\cite{bush_tropopause_1994}) \\

% phase velocity 8.7 m/s in RF25 paper??? actual speed of depression varies
% (four times higher stability compared to tropospheric  
% basic state equaled environmental state

% include meridional wind speed. Jet stream to the NE of fold (orientation NW-SE) 
% northward v component before depression, southward v component after

%%%% 3D %%%%%
\subsection{3D simulations of a propagating TD}
\label{sec:3D}

The first three-dimensional simulations are extending simulations 2D005 and 2D006 into the meridional dimension with a focus on the shape of the TD. As described in Section \ref{sec:propagation}, two mechanisms are able to describe the meridional propagation of GWs into the PNJ at 60$\degree$S. A non-parallel wave vector with respect to the background zonal wind and horizontal wind shear. Neglecting horizontal wind shear for simulations 3D007 and 3D008 allows the isolated investigation of a meridional propagation due to the orientation of the wave vector. In that manner, it is the goal to compare differences in wave propagation between an elongated N-S oriented depression, a N-S oriented, local depression (elliptic shape) and a tilted (NW-SE), local depression (simulation 3D008). The background wind for simulations 3D007 and 3D008 will be the same as in 2D005 and constant in meridional direction. Meridional winds only appear based on the Coriolis force, the environmental $V$ component is zero. 

Simulations 3D009 and 3D010 are addressing the second mechanism for horizontal wave propagation, the horizontal wind shear. In addition to the vertical wind shear, it is now the goal to design a baroclinic jet system with vertical and horizontal shear. Thus, the environmental $U$ component varies in horizontal and vertical direction, while the background meridional wind $V$ is still zero. Most likely, a reasonable wind field can be obtained by a two-dimensional Gaussian distribution. Simulation 3D009 is then planned to have the PNJ directly above the tropopause depression to provide a reference simulation for simulation 3D010 that finally has a baroclinic PNJ south of the tropopause depression. 

When introducing a baroclinic jet with horizontal and vertical wind shear as a background (environmental) wind field, its influence on the environmental state of the atmosphere has to be considered. The potential temperature distribution now has to balance the gradients in the wind field on the basis of the thermal wind relation
%
\begin{equation}
    \frac{\partial U}{\partial z} = -\frac{g}{\theta_0 f} \frac{\partial \theta}{\partial y}
    \label{equ:thermalWind}
\end{equation}
%
with $g$ and $f$ being the gravitational acceleration and Coriolis parameter and $\theta_0$ a constant basic state reference temperature. In the case of an isothermal stratosphere the pressure and density fields are fully defined, but further options might be investigated.

An additional simplification for 3D009 and 3D010 could be a barotropic jet with horizontal shear only instead of a baroclinic PNJ. Vertical wind shear would be zero, but the PNJ can still be positioned above the TD (3D009) and further to the south (3D010) to observe differences in the wave propagation with respect to the horizontal gradient in the wind.
% begin{equation}
%     u(\phi)=u_0 sin^3(\phi) cos(\phi)
% \end{equation}

%%%% Baroclinic instability %%%%
\subsection{Full simulation including troposphere}
\label{sec:barocInstability}

Though this final simulation (3D011) of the troposphere and stratosphere is most likely not covered within the proposed thesis, it is still mentioned at this point for the sake of completeness. It comprises the simulation of a baroclinic instability at the tropopause, as already carried out by \textcite{bush_tropopause_1994}, with the crucial difference of a PNJ in the upper stratosphere to the south. Though \textcite{bush_tropopause_1994} cover all relevant equations to define a balanced field that allows the initialisation of a baroclinic jet simulation, it is far from trivial to additionally consider a PNJ. 

 

\chapter{NOGWs excited by tropopause depressions in 3D}
\label{sec:results3D}
3D simulations of the previous section focused on the analysis of zonal and vertical properties of the tropopause and stratosphere with a low grid resolution in meridional direction. However, it is clear that in most cases the shape of a tropopause depression varies meridionally and the PNJ is not centered above the depression. Meridional processes are important and significantly influence the appearance of GWs in the upper stratopshere. Therefore, this section presents full 3D simulations with an identical resolution in streamwise and spanwise direction and with meridional gradients. \\
Subsequent to Chapter \ref{sec:resultsQ3D}, it is the goal to address the sensitivity of NOGWs above tropopause depressions to meridional properties of the tropopause and stratosphere by investigating the second research question
\begin{tcolorbox}[]
    (R2) How sensitive are NOGWs from propagating tropopause depressions to the depression's 3D shape and 3D properties of the stratospheric environment?
\end{tcolorbox}
In addition, realistic simulations of the stratosphere facilitate the investigation of the third research question, which refers to the zonally elongated phase lines in the horizontal cross sections of Figure \ref{fig:RF25_era5_horizonal}.
\begin{tcolorbox}[]
    (R3) Can NOGWs above propagating tropopause depressions explain the zonally elongated phase lines in ERA5 above the Southern Ocean?
    % the zonally elongated phase lines in the horizontal cross sections of Figure \ref{fig:RF25_era5_horizonal}. 
    % long vertical wave lengths in vertical cross section and elongated phase line in horizontal
\end{tcolorbox} 
All simulations of this chapter have the same temporal resolution (dt=\SI{60}{\second}) as simulations of Chapter \ref{sec:resultsQ3D}, but the horizontal and vertical domain is adapted to (n$_z$,n$_y$,n$_x$)=(251,480,720) grid points with a resolution of (dz,dy,dx)=(\SI{300}{\meter},\SI{20}{\kilo\meter},\SI{20}{\kilo\meter}). Again, periodic boundaries are used streamwise, linear sponge layers are used spanwise and the exponential sponge layer described in the "lessons learned" of Section \ref{sec:linear-MWs} is used vertically.

Two processes lead to a meridional propagation of GWs. The first process is solely caused by the 3D shape of the obstacle and becomes relevant as soon as the obstacle's shape changes in spanwise direction. The flow is not only deflected vertically, but also horizontally (flow around the isolated obstacle), which can result in a relevant lateral propagation of GWs. If, in addition, the shape is not symmetric in spanwise direction (e.g. a tilted tropopause depression with respect to the incoming flow), most of the GW activity might be channeled towards one side of the obstacle. Section \ref{sec:3D_noshear} analyses this process by comparing two simulations with different orientations of the tropopause depression in an environment without meridional wind shear. \\
The second process that leads to a meridional propagation of GWs is the modification of the wave-vector by meridional wind shear. Since this process is also linked to the depression's shape, Section \ref{sec:3D_shear} discusses variations of the background wind and Section \ref{sec:3D_shape} deals with variations of the depression's shape in the presence of meridional shear. \\
It follows a comparison of common methods to calculate the momentum flux from observations and from simulations in Section \ref{sec:3D_mf} and a short summary of the chapter in Section \ref{sec:3D_summary}.

% 3D orientation of wave-vector
% meridional propagation / meridional component of the wave vector can 

%%% No meridional shear %%%%
% \section{The influence of rotating the tropopause depression}
\section{The influence of the tropopause shape in the context of no meridional shear}
\label{sec:3D_noshear}
\begin{figure*}[t]
    \centering
    \includegraphics[width=0.99\textwidth]{figures_3D/waveletAna_overview_noShear.png}
    \caption{Horizontal cross sections at 40km above the tropopause for two simulations with no meridional shear as indicated by the purple arrows. Shown are $\Theta$', $\lambda_x$ and $\lambda_y$ after 72h. Dominant wavelengths at each grid point are based on a 1D wavelet analysis in all three dimensions and values below the \SI{95}{\percent} confidence level (with respect to red noise) are left out. The dotted lines in (a) and (d) represent the U-shaped pattern (Equation (\ref{equ:smith_parabola})) of GW activity derived by \textcite[]{smith_linear_1980} for linear MWs above a circular Witch of Agnesi mountain in a non-rotating fluid. The first row (a,b and c) shows a simulation with a tropopause depression oriented north-south, the second row (d,e and f) shows a simulation with a tilted depression.}
    \label{fig:waveletAna_noShear}
    % in a barotropic environment
\end{figure*}
This first section can be seen as a transition from simulations of the previous chapter with no meridional changes to simulations of the remaining chapter with meridional wind shear and meridional variations in the tropopause shape. It discusses two simulations without meridional wind shear, but with an elliptically shaped tropopause depression, so the depth of the tropopause changes spanwise in contrast to simulations of the preceeding chapter. In this way, the effect of the depression's meridional shape on the GW propagation can be isolated and it is not superimposed by the effect of meridional wind gradients.\\
Horizontal cross sections at $z$=\SI{40}{\kilo\meter} and $t$=\SI{72}{\hour} for both simulations are shown in Figure \ref{fig:waveletAna_noShear}. The background wind is similar to the yellow simulation in Figure \ref{fig:q3D_wind} with a $u_{PNJ,max}$=\SI{100}{\meter\per\second} and no meridional gradients as indicated by the purple arrows in (a) and (d). In the upper simulation of Figure \ref{fig:waveletAna_noShear} (a,b and c), the zonal half width of the cosine tropopause depression is again $L_x$=\SI{300}{\kilo\meter} and the meridional one is twice as wide with $L_y$=\SI{600}{\kilo\meter}. In the lower simulation (d,e and f) this tropopause depression is rotated horizontally by \SI{45}{\degree}, so the longer axis of the depression is oriented north-west to south-east (NW-SE). Dashed lines in Figure \ref{fig:waveletAna_noShear} outline the shape of the depression and clarify the orientation. \\
The observed GW pattern in the simulation with a north to south (N-S) oriented depression coincides with the theory of \textcite[]{smith_linear_1980}. The main GW activity aloft mimics a parabola (U-shape) that departs from the center and extends into the lee of the tropopause depression. \textcite[]{smith_linear_1980} neglected rotational effects and derived his linear solutions for GWs above an isolated circular Witch of Agnesi mountain (Equation (\ref{equ:agnesi})) and for a constant wind profile. He states that his results should be applicable to scales from 5 to \SI{50}{\kilo\meter}, but Figure \ref{fig:waveletAna_noShear} demonstrates that it is still a good approximation for variations from the idealized setup and for larger scales that are subject to rotational effects. The thick dotted lines in (a) and (d) represent the solution of Smith's parabola
\begin{equation}
    y^2 = \frac{L \, N}{U} z \, x
    \label{equ:smith_parabola}
\end{equation}
for $L$=\SI{300}{\kilo\meter}, $U$=\SI{100}{\meter\per\second}, $N$=\SI{0.02}{\per\second} and $z$=\SI{40}{\kilo\meter}. The meridional asymmetry of the GW perturbations in (a) is caused by the Coriolis force with slightly stronger perturbations in the north ($f$ of simulations is negative for southern hemisphere), but the general picture in (a) fits to the U-shape described by \textcite[]{smith_linear_1980} for GWs aloft. \\
In the idealized linear setup of \textcite[]{smith_linear_1980} phase lines along the parabola always point back towards the GW source, so the wave's orientation changes proportional to the distance from the source region. In the simulation with a N-S oriented tropopause depression ((a),(b) and (c) in Figure \ref{fig:waveletAna_noShear}) this effect is weak, but noticeable. (b) and (c) depict zonal ($\lambda_x$) and meridional ($\lambda_y$) wavelengths from a 1D wavelet analysis (Section \ref{sec:wavelet}) and, in particular, $\lambda_x$ increases along the parabola and indicates a turning of the horizontal wave-vector. $\lambda_y$ is too large in the proximity of the tropopause fold (besides some values in the top right corner of (c)) and, in this context, phase lines rather point towards a leeward area of the fold and not towards its center, but this discrepancy is plausible and likely results from the elliptic (not circular) shape of the depression.

The effect of tilting the tropopause depression with respect to the zonal flow is apparent in (d),(e) and (f) of Figure \ref{fig:waveletAna_noShear}. A NW-SE oriented depression channels the GW activity southward, though the Coriolis effect counteracts. The shortest wavelenghts are now located south of the depression (bluish colors in (e)) and, again, $\lambda_x$ increases along the parabola while values for $\lambda_y$ are only significant far south-east of the depression in (f). Phase lines along the parabola in (d) tend to point closer to the center of the depression compared to the first simulation in (a), because the difference between the streamwise and spanwise extend of the depression is decreased by the horizontal rotation. \\
Overall, the 3D shape of an obstacle (e.g. a tropopause depression) can influence the propagation direction of the excited GWs and simply rotating the tropopause depression in an environment without meridional wind shear had a significant effect on the GW activity in the upper stratosphere. How this rotation influences the GWs in a realistic scenario with meridional gradients in the background wind will be discussed in Section \ref{sec:3D_shape}. 

% \cite[]{smith_linear_1980}
% $Fr=\frac{U}{h_m \, N} \approx 2$
% shall see that the small-amplitude theory describes the tendency of the flow to be diverted around the mountain but that this theory is valid only for large Froude number F
% All three of the qualitative features; the radial, outward tilting phase lines, the concentration of the disturbance along parabolas which widen with height, and the decay away from the mountain,
% along parabolas which widen with height z, and the decay away from the mountain, are evident 

% The obstacle feels closer to a circular shape for the wind""""
% deflected laterally!
% in the presence of gradients in meridional wind

%%% PNJ with meridional shear %%%
\section{The influence of meridional shear}
\label{sec:3D_shear}
Let us recall the equation for the temporal evolution of the meridional wavenumber from linear ray-tracing
\begin{equation}
    \frac{dl}{dt} = -(k \frac{\partial U}{\partial y} + l \frac{\partial V}{\partial y} + \frac{\beta f}{\hat{\omega}})
    \approx -k \frac{\partial U}{\partial y}
    \label{equ:meridionalRefraction2}
\end{equation}
mentioned in the introduction (e.g. \cite[]{dunkerton_inertiagravity_1984} or \cite[]{eckermann_ray-tracing_1992}). In our simulations $f$ is constant and the background flow is purely zonal, so $\beta=0$ and $\frac{\partial V}{\partial y}=0$ lead to the approximation in Equation (\ref{equ:meridionalRefraction2}). For conservative wave propagation and a stationary background flow $U$, the ground-based frequency $\omega$ is conserved along a ray ($\frac{d \omega}{d t} = 0$) and 
\begin{equation}
    \omega = k c_{P,x} + l c_{P,y} = \textrm{const.}
    \label{equ:const_omega}
\end{equation}
with $c_{P,x}$ and $c_{P,y}$ being the components of the ground-based horizontal phase speed (\cite[]{lighthill_waves_1978} and \cite[]{eckermann_ray-tracing_1992}). Therefore, modifications of $l$ due to the background shear (Equation (\ref{equ:meridionalRefraction2})) always imply a change of $k$ and a turning of the GW. Linear theory also shows that the direction of the horizontal wave vector ($k$,$l$) and of the intrinsic horizontal group velocity are equal for upward propagating GWs (\cite[]{sato_origins_2009}). Under the assumption of a purely zonal background flow the sign of $l$ is equal to that of the ground-based meridional group velocity and represents the direction of wave propagation.\\
The approximation on the right hand side of Equation (\ref{equ:meridionalRefraction2}) shows two terms that are relevant for the generation of a meridional wave component $l$, the meridional shear $\frac{\partial U}{\partial y}$ and the zonal wavenumber $k$. Variations of the zonal wavenumber $k$ or wavelength $\lambda_x$ are closely related to the zonal shape of the tropopause depression and will be discussed in the next section. At first, this section focuses on meridional variations of the background wind.

\begin{figure*}[tbp]
    \centering
    \includegraphics[width=0.99\textwidth]{figures_3D/3D-th-referenceSim.png}
    \caption{The reference simulation of the entirely 3D simulations with meridional background wind shear. (a),(c) and (e) show horizontal cross sections of $\Theta'$ at z=\SI{40}{\kilo\meter} for three timestamps. (b),(d) and (f) show corresponding meridional cross sections \SI{900}{\kilo\meter} in the lee of the propagating tropopause fold. The position is also indicated in the horizontal cross sections by the dashed black lines. The purple lines in (a),(c) and (e) refer to the location of the PNJ. Meridional cross sections also show zonal wind $u$ (dashed lines) and isentropes (solid lines).}
    \label{fig:3D-reference}
\end{figure*}
\begin{figure*}[tbp]
    \centering
    \includegraphics[width=0.96\textwidth]{figures_3D/waveletAna_angle.png}
    \caption{Quite similiar to Figure \ref{fig:waveletAna_noShear}, but showing simulations with meridional wind shear as visualized in Figure \ref{fig:wind_profs}b. Here, $\lambda_y$ moved to the second column and the third column presents the angle $\phi=\arctan(\frac{\lambda_y}{\lambda_x})$ between the phase lines and the x-axis. Again, each row refers to a different simulation and the second and third column is based on a 1D wavelet analysis in each dimension. The corresponding zonal wavelength $\lambda_x$ and the vertical wavelength $\lambda_z$ are plotted in Figure \ref{fig:waveletAna_xz} in Appendix \ref{appA}. The third row ((g),(h) and (i)) is the reference simulation presented in Figure \ref{fig:3D-reference} and labels in each row describe the respective variation. The horizontal purple lines indicate the center of the PNJ.}
    % Hashed areas in the  below \SI{30}{\degree}. For clarification $\phi$ and $\lambda_z$ are plotted in Figure \ref{fig:waveletAna_angle} in Appendix \ref{appA}.
    % Hashed areas in the third column indicate vertical wavelengths larger than \SI{20}{\kilo\meter} at z=\SI{40}{\kilo\meter}.
    \label{fig:waveletAna}
    % between 6 and \SI{20}{\kilo\meter}
\end{figure*}
Figure \ref{fig:3D-reference} and \ref{fig:waveletAna} provide the basis for the following discussion. The first one shows the evolution of the reference simulation for the scenario with meridional wind shear. The center of the PNJ is \SI{1500}{\kilo\meter} south of the tropopause depression centered at $y=0$, because this represents a common scenario above the Southern Ocean west and south of Australia with a PNJ at approximately \SI{60}{\degree S} and Rossby wave trains with tropopause folds further to the north between 30-\SI{55}{\degree S} (\cite[]{skerlak_tropopause_2015}). The 2D wind profile in the meridional cross sections ((b),(d) and (f)) is the product of the vertical wind profile used in the previous section and a meridional distribution centered at $y=\SI{-1500}{\kilo\meter}$ (red curve in Figure \ref{fig:wind_profs}). Similar to the preceding section, the widths of the depression are $(L_x,L_y)=(\SI{300}{\kilo\meter},\SI{600}{\kilo\meter})$ and the tropopause propagates zonally with a constant speed $c_{tf}=\SI{13.88}{\meter\per\second}$. \\
In Figure \ref{fig:3D-reference} GWs excited above the tropopause depression are apparently refracted southward into the PNJ and develop a significant meridional wave-vector component $l$, so phase lines in the horizontal cross sections turn counterclockwise. The wavelet analysis in Figure \ref{fig:waveletAna} confirms this impression. The third row ((g),(h) and (i)) belongs to the reference simulation and, clearly, $\lambda_y$ is smaller south-east of the depression where the GW has propagated into the PNJ. In the same area, the angle $\phi=\arctan(\frac{\lambda_y}{\lambda_x})$ between the phase lines and the x-axis is also smaller and implies a rotation of the wave vector. The corresponding $\lambda_x$ is visualized in Figure \ref{fig:waveletAna_xz} in Appendix \ref{appA} and is larger close to the PNJ. These observations are entirely consistent with linear ray-tracing theory and Equations (\ref{equ:meridionalRefraction2}) and (\ref{equ:const_omega}). MW-like GWs propagate against the background flow and imply a negative $k$. Moreover, the southward shift of the PNJ with respect to the GW source in the center of the domain leads to a negative $\frac{\partial U}{\partial y}$ north of the PNJ and hence $l$ increases in magnitude, but with a negative sign. A negative meridional wave vector component develops and the GW is refracted southward into the PNJ. It is clear that this process is not unlimited. As $l$ increases, $k$ decreases (Equation (\ref{equ:const_omega})) and starts to dampen the further development of a meridional wave component due to $\frac{\partial U}{\partial y}$ in Equation (\ref{equ:meridionalRefraction2}). \\
A PNJ centered above the tropopause depression as in (a),(b) and (c) of Figure \ref{fig:waveletAna} completely alters the GW pattern. Here, $\frac{\partial U}{\partial y}$ is negative in the north of the depression and positive in the south. GWs that propagate to the north (south) develop a negative (positive) $l$ and are refracted back towards the center of the domain. Again, GWs propagate towards the maximum wind speed of the PNJ, but in this scenario the generation of a meridional wave component $l$ is suppressed and phase lines stay more or less orthogonal to the zonal backgroud flow. \\ % as depicted in (a) and (b) of Figure \ref{fig:waveletAna}
In the simulation of (m),(n) and (o) (row five of Figure \ref{fig:waveletAna}) the maximum wind speed of the PNJ is increased from $U_{PNJ,max}=\SI{100}{\meter\per\second}$ to $U_{PNJ,max}=\SI{130}{\meter\per\second}$. Signs for $k$ and $\frac{\partial U}{\partial y}$ are similar to the reference simulation, so GWs also propagate southward. However, the development of a negative $l$ and the corresponding turning of the wave is stronger, because the absolute value of $\frac{\partial U}{\partial y}$ is larger. Comparing (h) and (i) to (n) and (o) in Figure \ref{fig:waveletAna} supports this statement. $\lambda_y$ in (n) is generally smaller than in (h) and $\phi$ in (o) indicates a trend towards more horizontal phase lines with smaller $\lambda_y$ (wide purple area north of the PNJ), too.\\
We note that GW phase lines in the horizontal cross section at z=\SI{40}{\kilo\meter} (Figure \ref{fig:waveletAna}m) appear quite similar to the zonally elongated phase lines in the ERA5 data (Figure \ref{fig:RF25_era5_horizonal}) observed by \textcite[]{dornbrack_stratospheric_2022} and addressed in research question (R1). Strong winds in the upper stratosphere and the southward position of the PNJ seem to be crucial, if not necessary, features to observe such a GW pattern. Either way, these conditions are likely in the southern hemisphere and propagating tropopause depressions potentially excite those GWs with zonally elongated phase lines. The influence of the tropopause shape in the presence of meridional wind shear is discussed in the next section.

% Another view on this refraction of GWs into the PNJ are critical levels... ??
% ground-based horizontal phase speed is always parallel to K 
% This lies the foundation for the following discussion.
% relative to the mean wind = intrinsic
% A = E/\omega$ = const.$  is the wave action 
% $ \omega = \hat{\omega} + k U = const.$ 
% beta = df/dy so beta plane approximation, usually also gradient of f changes on Earth
% trailing GWs 

\section{The influence of the tropopause shape in the context of meridional shear}
\label{sec:3D_shape}
%As stated in the previous section, 
The zonal wavenumber $k$ in Equation (\ref{equ:meridionalRefraction2}) is closely linked to the zonal width of the tropopause depression. A depression with a smaller width excites GWs with a smaller horizontal scale (larger $k$) and consequently, the growth of a meridional wave component $\frac{dl}{dt}$ in the presence of meridional wind shear is greater. The second (fourth) row of Figure \ref{fig:waveletAna} displays a simulation with a wider (narrower) tropopause depression as the reference simulation. Comparing these three simulations confirms the ray-tracing theory. The smaller the depression's zonal width, the smaller $\lambda_x$ ((e),(h) and (k) in Figure \ref{fig:waveletAna_xz}) and the smaller $\lambda_y$ ((f),(i) and (l)in Figure \ref{fig:waveletAna}) in the upper stratosphere. In Chapter \ref{sec:resultsQ3D}, we also concluded that the GW activity in the stratosphere is inversely proportional to the width of the tropopause depression. Figure \ref{fig:waveletAna_mf} in Appendix \ref{appA} further substantiates this fact and emphasizes that the zonal ((e),(h) and (k)) and the meridional ((f),(i) and (l)) momentum flux increases for smaller depression widths. \\
A decreasing depression width implies decreasing GW scales and an increase in the meridional momentum flux, but does it also influence the orientation of the wave vector? In this context, we again refer to the angle $\phi$ at $z=\SI{40}{\kilo\meter}$ in the third column of Figure \ref{fig:waveletAna}. Though variations of the depression width are high between simulations, there is no clear trend of the ratio $\frac{\lambda_y}{\lambda_x}$ or the orientation of the wave vector ((f),(i) and (l)). All three simulations exhibit a similar pattern with rather spanwise oriented phase lines above and in the lee of the depression, and more streamwise oriented phase lines closer to the PNJ where the $\frac{\partial U}{\partial y}$ increases. But a trend towards smaller $\phi$, as for the stronger PNJ in row 5 of Figure \ref{fig:waveletAna}, is not observable.

\begin{figure*}[t]
    \centering
    \includegraphics[width=0.99\textwidth]{figures_3D/3D-EF-MF.png}
    \caption{Similar cross sections as Figure \ref{fig:3D-reference} with horizontal cross sections at z=\SI{40}{\kilo\meter} in (a) and (c) and meridional cross sections in (b) and (d) for x=\SI{9400}{\kilo\meter} showing meridional momentum flux MF$_y$ instead of $\Theta'$. The first row is again the 3D "reference" simulation at $t=\SI{72}{\hour}$, the second row is a simulation with identical settings, but a NW-SE oriented tropopause depression. In other words, the depression is rotated horizontally by 45°. Black arrows in (a) and (c) illustrate the direction and relative value of the horizontal energy flux \textbf{EF}. The reference simulation is the third row and the simulation with a rotated depression is represented by the last row in Figure \ref{fig:waveletAna}.}
    \label{fig:3D-MFy}
\end{figure*}
The last row in Figure \ref{fig:waveletAna} presents a simulation with a rotated tropopause depression as in Section \ref{sec:3D_noshear}, but in the presence of meridional wind shear. Interestingly, $\lambda_y$ and $\phi$ are larger for the NW-SE oriented depression, so rotating the depression counteracts the development of $l$ and zonally elongated phase lines. It seems counterintuitiv, because the orientation of the depression already leads to a small meridional wave vector component in the no-shear case of Section \ref{sec:3D_noshear}. However, the wider width in the direction of the background flow reduces $k$ and the growth of $l$ and, therefore, constrains the refraction and meridional propagation of the GW. Figur \ref{fig:3D-MFy} further illustrates this fact. The meridional momentum flux MF$_y$ is lower for the simulation with a tilted depression in the horizontal cross section (c) and the meridional cross section (d) compared to (a) and (b) for a N-S oriented depression with an identical size. Vectors of the energy flux \textbf{EF} in (a) and (c) clarify the southward propagation of the GWs into the PNJ and the intrinsic propagation against the prevailing wind for both simulations, but, again, \textbf{EF} is slightly greater for the N-S oriented depression.

Overall, the meridional momentum and energy fluxes significantly increase for smaller zonal widths of the tropopause depression, but $\phi$ or the zonally elongated phase line pattern in the upper stratosphere is not directly sensitive to the depression's zonal width. On the other hand, $\phi$ seems to be linked to the depression's orientation, because a NW-SE oriented depression results in a wider $\phi$ of the phase lines with the x-axis. This might be linked to its larger width in the direction of the background flow, but additional factors have to be relevant, too, because we just stated that $\phi$ is not sensitive to the zonal width.

% general statement on momentum fluxes larger for smaller scale GWs

\section{Momentum flux calculation from an observational perspective}
\label{sec:3D_mf}
In numerical simulations the vertical flux of horizontal momentum (MF) can be calculated directly from available wind perturbations ($u'$,$v'$ and $w'$) by utilizing equation \ref{equ:mf}. It was used in the preceding sections, too. However, most satellite or ground-based instruments that provide observations of the stratosphere and MLT on a reasonable scale to study GWs are only capable of measuring temperature (e.g. \cite{wu_satellite_1996}, \cite[]{ern_absolute_2004}, \cite{hindley_gravity_2019}, \cite{kaifler_compact_2021}). Temperature perturbations alone allow the calculation of potential energy distributions, but usually momentum flux is the variable of interest to investigate wave-mean flow interactions in the middle atmosphere and constrain GW parameterizations in climate simulations (\cite[]{geller_comparison_2013} or \cite[]{kim_overview_2003}). \\
In this context, \textcite[]{ern_absolute_2004} derived a formulation of MF that depends on the temperature amplitude $\hat{T}$ and the wave's horizontal and vertical wavelength instead of $\overbar{u'w'}$. They start with the expression of the vertical flux of horizontal pseudomomentum valid for conservative wave propagation
\begin{equation}
    (\mathrm{MF}_x, \mathrm{MF}_y) = \bar{\rho} (1-\frac{f^2}{\hat{w}^2}) \Bigl(\overbar{u'w'},\overbar{v'w'}\Bigr) 
    % \approx \bar{\rho}  (\overbar{u'w'},\overbar{v'w'})
    \label{equ:ps-mf}
\end{equation}
(\cite[]{fritts_gravity_2003}). We follow their convention and refer to it as momentum flux MF with MF$_x$ and MF$_y$ being the zonal and meridional MF respectively. They utilize the generally valid linear polarization relations (Equation (14) in \textcite[]{fritts_gravity_2003}) and the dispersion relation
\begin{equation}
    \hat{w}^2 = \frac{N^2 \Bigl(k^2+l^2\Bigr) + f^2 \Bigl(m^2 + \frac{1}{4H^2}\Bigr)}{k^2+l^2+m^2+\frac{1}{4H^2}}
    \label{equ_lid:dispersion_noAss}
\end{equation}
from \textcite[]{fritts_gravity_2003} to end up with
\begin{equation}
    (\mathrm{MF}_x, \mathrm{MF}_y) = A \cdot B \frac{\bar{\rho}}{2} \Bigl(\frac{g}{N}\Bigr)^2 {\Bigl(\frac{\hat{T}}{\bar{T}}\Bigr)^2} \Bigl(\frac{k}{m},\frac{l}{m}\Bigr).
    \label{equ:mf-tp}
\end{equation}
The only assumptions that accompany equation \ref{equ:mf-tp} are a monochromatic wave perturbation and hydrostatic equilibrium. In \textcite[]{ern_absolute_2004} and many follow-up publications (e.g. \cite[]{preusse_characteristics_2014}, \cite[]{ern_gracile_2018}, \cite[]{hindley_gravity_2019}) the factors $A$ and $B$ are neglected by refererring to the midfrequency approximation. NOGWs above tropopause depression can be considered low frequency waves with large horizontal wavelengths. It needs to be checked if $A$ and $B$ are still neglible in this spectral range. While the factor
\begin{equation}
    \begin{split}
        A = &\biggl(1-\frac{\hat{\omega}^2}{N^2}\biggr) \biggl(1 + \frac{1}{m^2} \Bigl(\frac{1}{2H}-\frac{g}{c_s^2}\Bigr)^2\biggr)^{-1} \\
            &\biggl(1+\Bigl(\frac{f}{m \hat{\omega}}\Bigr)^2 \Bigl(\frac{1}{2H} - \frac{g}{c_s^2}\Bigr)^2\biggr)^{0.5}
    \end{split}
    \label{equ:A}
\end{equation}
naturally appears following the approach of \textcite[]{ern_absolute_2004}. The factor
\begin{equation}
    B = \left| \frac{\tilde{\Theta}}{\tilde{T}} \right|^2 = \left| 1 + \frac{1}{\beta} \frac{\gamma-1}{c_s^2} \frac{\hat{\omega}}{\frac{N^2}{g}}i\right|^{-2}
    \label{equ:B}
\end{equation}
with
\begin{equation}
    \beta = -\frac{\hat{\omega}}{N^2-\hat{\omega}^2} \biggl(m + \Bigl(\frac{1}{2H}-\frac{g}{c_s^2}\Bigr)i\biggr)
    \label{equ:beta}
\end{equation}
represents the error by assuming equal amplitudes for $\hat{\Theta}$ and $\hat{T}$ considering all motions to be adiabatic (\cite[]{fritts_gravity_2003} and \cite[]{ern_directional_2017}).
\begin{wrapfigure}{r}{7.5cm}
    \includegraphics[width=7.5cm]{figures_3D/waveletAna_mfcorrection_factor.png}
    \caption{The correction factor A$\cdot$B in equation \ref{equ:mf-tp} for a range of vertical and horizontal wavelengths. It is reproduced from \textcite[]{ern_directional_2017} (supporting information) for a larger $f$ and smaller $c_s$. The dashed rectangle contains all combinations of wavelenghts from the wavelet analysis.}
    \label{fig:mf_correction}
\end{wrapfigure}
In their supporting information \textcite[]{ern_directional_2017} visualize $A$ and $B$ for a common range of vertical and horizontal wavelengths and typical stratospheric values of $N=\SI{0.02}{\per\second}$, $H=\SI{7}{\kilo\meter}$ and $T=\SI{250}{K}$ resulting in $c_s = \SI{316}{\meter\per\second}$. Furthermore, $\gamma=0.4$ and $g=\SI{9.81}{\meter\per\second^2}$. The Coriolis parameter $f$ was chosen for a latitude of 30°. All values are representative for the idealized simulations of this work, but $f$ was a factor of 1.64 larger in the simulations and the background temperature of the isothermal atmosphere was $T=\SI{239}{K}$. Therefore, Figure (S1c) of \textcite[]{ern_directional_2017} is reproduced in Figure \ref{fig:mf_correction} for a Coriolis parameter $f=\SI{1.2e-4}{\per\second}$ (latitude of 55°) and $c_s = \SI{310}{\meter\per\second}$ due to $T=\SI{239}{K}$. \\
Adapting the Coriolis parameter had no noticeable effect. Lowering $c_s$ made $A \cdot B$ smaller, so less negligible, but the factor is still greater than $0.95$ for most combinations of wavelengths appearing in the simulations (dashed rectangle). In general, Figure \ref{fig:mf_correction} clarifies that $A \cdot B$ is much more significant for non-hydrostatic or high frequency waves, while GWs with large horizontal scales are less affected and $A \cdot B$ can be neglected for the following comparison.

At this point, it is important to clarify another relation. \textcite[]{ern_absolute_2004} derived an equation for the momentum flux that depends on the wave's temperature amplitude $\hat{T}$. $\hat{T}$ results from the spectral analysis of the 3D temperature measurements just like horizontal and vertical wavelengths and refers to the amplitude of a perfect sine or cosine wave. These spectral analysis methods improved consistently over the past two decades. Examples are the "S3D" method by \textcite[]{lehmann_consistency_2012} or the work of \textcite[]{wright_exploring_2017} who extended the Stockwell transform to 3D and applied it to new satellite measurements with AIRS (see also \cite[]{hindley_gravity_2019} and \cite[]{hindley_18year_2020}). When following these analysis methods it makes sense to use $\hat{T}$ for calculating MF, because maximum amplitudes might be missed by the coarse resolution of the measurements. In contrast, numerical models provide temperature perturbations $T'$ on a regular grid and it is possible to calculate MF directly via the temperature variance $\overbar{T'^2}$. Again, the overbar denotes averaging over one or multiple full wave cycles. Relating $\overbar{T'^2}$ to $\hat{T}^2$ yields
\begin{equation}
    \overbar{T'^2} = \overbar{\hat{T} sin(\phi)^2} = \frac{1}{2}\hat{T}^2,
    \label{equ:tvariance}
\end{equation}
because $\overbar{sin(\phi)^2} = \frac{1}{2}$. Since temperature variance defines the eddy potential energy per unit mass $E_p$ (first row in equation \ref{equ:epot}), we can use equation \ref{equ:tvariance} to rewrite $E_p$ in terms of $\hat{T}$
\begin{equation}
    \begin{split}
        E_p &= \frac{1}{2} \Bigl(\frac{g}{N}\Bigr)^2 \overbar{\Bigl(\frac{T'}{\bar{T}}\Bigr)^2} \\
            &= \frac{1}{4} \Bigl(\frac{g}{N}\Bigr)^2 \Bigl(\frac{\hat{T}}{\bar{T}}\Bigr)^2
    \end{split}
    \label{equ:epot}
\end{equation}
and relate it to the momentum flux
\begin{equation}
    (\mathrm{MF}_x, MF_y) = \bar{\rho} \Bigl(\frac{g}{N}\Bigr)^2 \overbar{\Bigl(\frac{T'}{\bar{T}}\Bigr)^2} \Bigl(\frac{k}{m},\frac{l}{m}\Bigr) = 2 \bar{\rho} E_p \Bigl(\frac{k}{m},\frac{l}{m}\Bigr)
    \label{equ:mf-epot}
\end{equation}
based on equation \ref{equ:mf-tp} and neglecting the factor $A \cdot B$ (\cite[]{ern_gracile_2018}, \cite*[]{ern_intermittency_2022}). From Figure \ref{fig:mf_correction} it is clear that the factor $A \cdot B$ is neglible for the low frequency GWs excited above tropopause folds, so in the following  we will use equation \ref{equ:mf-epot} to calculate the pseudomomentum flux from temperature perturbations. In a similar manner it is possible to show that the factor $\bigl(1-\frac{f^2}{\hat{\omega}^2}\bigr)$ in equation \ref{equ:ps-mf} can be neglected to calculate the pseudomomentum flux from wind perturbations. Again, the factor is greater than 0.95 for most combinations of horizontal and vertical wavelengths that appear in the idealized simulations, so the vertical flux of horizontal pseudomomentum is underestimated by about \SI{5}{\percent} or less when replaced by the conventional momentum flux $\bar{\rho}(\overbar{u'w'},\overbar{v'w'})$ (equation \ref{equ:mf}) under the midfrequency approximation ($N >> \hat{\omega} >> f$). \\
For a broader understanding we can obtain another perspective on this error by introducing the total energy
\begin{equation}
    E_0 = E_k + E_p = \frac{1}{2} \Bigl(\overbar{u'^2} + \overbar{v'^2} + \overbar{w'^2}\Bigr) + \frac{1}{2} \Bigl(\frac{g}{N}\Bigr)^2 \overbar{\Bigl(\frac{T'}{\bar{T}}\Bigr)^2}
    \label{equ:etot}
\end{equation}
with the kinetic energy per unit mass $E_k$ (e.g. \cite[]{gill_atmosphere-ocean_1982} or \cite[]{tsuda_global_2000}). For non-rotating GWs the kinetic and potential energy tend to be the same ($ E_k \approx E_p$). In that case, the wave energy is equipartiioned and from equation \ref{equ:mf-epot} it follows
\begin{equation}
    \mathbf{MF} = 2 \bar{\rho} E_p \frac{\mathbf{K}}{m} = \bar{\rho} E_{0} \frac{\mathbf{K}}{m}
    \label{equ:mf-etot}
\end{equation}
with the horizontal wavenumber vector $\mathbf{K}$ (compare to \cite[]{andrews_wave-action_1978} or \cite[]{fritts_gravity_2003}). The kinetic energy part becomes more and more dominant for low frequency waves (\cite[]{gill_atmosphere-ocean_1982}), so the error of calculating the pseudomomentum flux under the midfrequency approximation (equation \ref{equ:mf} and \ref{equ:mf-epot}) is proportional to $\frac{E_k}{E_p}$.  

After this extensive discussion on relevant approximations and relations, Figure \ref{fig:mf_scatter} finally compares zonal and meridional MF from wind perturbations to MF from temperature perturbations for all 3D simulations in Figure \ref{fig:waveletAna}.
\begin{figure*}[t]
    \centering
    \includegraphics[width=0.99\textwidth]{figures_3D/waveletAna_mf_scatter.png}
    \caption{Scatter plots of the zonal (a) and meridional (b) MF at z=\SI{40}{\kilo\meter} after 72h for all simulations in Figure \ref{fig:waveletAna}. The x-axis refers to the MF calculated from temperature perturbations, the y-axis refers to the MF calculated from wind perturbations. Colored dashed lines are linear fits with slope m and intercept $y_{0}=0$ for each individual simulation.}
    \label{fig:mf_scatter}
\end{figure*}
Overall, momentum fluxes from wind and temperature correlate well. R values are high for all simulations, so linear regressions in Figure \ref{fig:mf_scatter} represent meaningful relations. Especially linear fits of the zonal MF show a good agreement with slope values $m \approx 1$ with one exception. For reasons not fully understood, the slope of the linear regression for the simulation with the PNJ centered above the propagating tropopause depression has a slope of $m \approx 1.7$. In this simulation setup GWs propagate upward into the PNJ without a turning of phase lines resulting in no meridional flux. Nevertheless, we would expect the correlation of the MF$_x$ to be similar to the remaining simulations. \\
MF$_y$ (Figure \ref{fig:mf_scatter}b) also shows good agreement, but a general low bias of temperature-based with respect to wind-based MF can be observed. Only the simulation for the smallest depression width $L=\SI{200}{\kilo\meter}$ has an almost perfect slope while the general picture indicates $15-\SI{25}{\percent}$ higher values for MF$_y$ from winds.\\
MF$_x$ and MF$_y$ suggest a higher correlation for smaller depression widths $L$. Three simulations ($L=400,300,\SI{200}{\kilo\meter}$) are not enough to be conclusive here, but this trend could be related to the effect of rotation that becomes more relevant for wider tropopause folds that result in larger horizontal wavelengths and lower intrinsic frequencies $\hat{\omega}$. As discussed in the beginning of this section, the neglected factor $A \cdot B$ for the MF from $T'$ is not sensitive to large horizontal scales or an increased impact of Coriolis. On the other hand, the factor $\bigl(1-\frac{f^2}{\hat{\omega}^2}\bigr)$ in equation \ref{equ:ps-mf} decreases for smaller $\hat{\omega}$, so MF from wind is overestimated, when this factor is neglected for low frequency waves with $\hat{\omega}$ approaching $f$. Maybe it is this simplification under the midfrequency approximation that becomes less valid and leads to a higher MF from winds for wider tropopause folds.

Despite these uncertainties, we conclude that the calculation of MF from temperature perturbations still leads to meaningful results for the spectrum of GWs excited by propagating tropopause depressions. Considering the uncertainty of real measurements the correspondence between temperature-based and wind-based MF is very good and uncertainties due to the calculation are tolerable. In addition, it shows that the majority of the GWs within the idealized simulations obey the polarization and dispersion relations of GWs and propagate linearly up to an altitude of at least \SI{40}{\kilo\meter}. The analysis of horizontal cross sections at lower levels leads to similar conclusions and is therefore left out.

% zero intercept was used, so larger fluxes might have stronger influence slope value??
% A particularly striking result is a widespread 

% The right part in equation \ref{equ:mf-etot} fits to the work of \textcite[]{andrews_wave-action_1978} or the work of \textcite[]{fritts_spectral_1993} who derived energy spectra in the upper atmosphere from observations. 

% Vertical timeseries of temperature from Lidar observations usually aren't sufficient to derive horizontal momentum fluxes without additional information or assumptions for horizontal wavelengths.
% Gracile paper (Ern 2018) states a factor of Ekin/Epot = 5/3???? should be E0/Epot??
% $f$ is the inertial frequency
% p=5/3 (intrinsic frequency ωˆ spectrum of the GW wave energy density assumed to decrease with ωˆ−5/3)
% The acceleration or deceleration (X, Y ) of the background flow, in the following for simplification called gravity wave drag, is given by the vertical gradient of momentum flux: (X, Y ) = − 1 % ∂ (Fpx, Fpy ) ∂z , (12) with X and Y the drag in the zonal and meridional directions, respectively, and z the vertical coordinate.
 
% F = rho * k/m * Epot,max.
% Es gilt also Etot = Ekin + Epot = Epot,max (in Worten... zum Zeitpunkt, wo die potentielle Energie über die Schwingung maximal wird verschwindet der kinetische Anteil --> Ekin = 0). Wenn wir nun F allein aus den Temperaturvarianzen berechnen wollen nehmen wir an, dass
% Epot {gemittelt über eine Wellenperiode} = Ekin {gemittelt über eine Wellenperiode} (hier kommen dann wohl Annahmen wie linear und non-dissipative mit rein). Daraus folgt dann 
% --> Etot = 2 * Epot {gemittelt über eine Wellenperiode}
% --> F = rho * k/m * 2*Epot {gemittelt über eine Wellenperiode} = rho * k/m *(g/N)^2 * average((T'/T)^2)

\section{Summary and answer to research question (R2) and (R3)}
\label{sec:3D_summary}
This chapter mainly presented 3D simulations of the stratosphere with identical resolutions in streamwise and spanwise direction, with meridional gradients of the lower boundary (tropopause) and with vertical and meridional gradients in the zonal background flow. It is the most realistic setup within the framework of this thesis and only a coupling of tropospheric and stratospheric airflows with a baroclinic instability at tropopause level and a PNJ in the upper stratopshere would be closer to reality. \\
As an extension to the 2D sensitivity study in Chapter \ref{sec:resultsQ3D}, this chapter focused on the sensitivity with respect to meridional properties of the tropopause and stratosphere and addressed the second research question.
\begin{tcolorbox}[]
    (R2) How sensitive are NOGWs from propagating tropopause depressions to the depression's 3D shape and 3D properties of the stratospheric environment?
\end{tcolorbox}
In the context of no meridional wind shear (Section \ref{sec:3D_noshear}), the orientation of the tropopause depression can significantly alter the GW activity and channel GWs in one lateral direction. Tilting the depression with respect to the incoming flow increases the meridional propagation of GWs though the wave vector is only marginally modified. In the presence of meridional wind shear, rotating the depression has the opposite effect and decreases meridional momentum and energy fluxes, because the depression's width increases. This is consistent with the conclusion on the depression's zonal width in Chapter \ref{sec:resultsQ3D}. Overall, meridional momentum and energy fluxes in the upper stratosphere
\begin{itemize}
    \item are inversely proportional to the zonal width of the tropopause depression.
    \item decrease for a horizontally rotated tropopause depression (Figure \ref{fig:3D-MFy}).
    \item are proportional to the strength (wind speed) of the PNJ.
    \item are highly dependent on the relative meridional position of the PNJ. A PNJ directly above the depression minimizes the meridional propagation of GWs.
\end{itemize}
This chapter also focused on answering the third research question.
\begin{tcolorbox}[]
    (R3) Can NOGWs above propagating tropopause depressions explain the zonally elongated phase lines in ERA5 above the Southern Ocean?
    % the proposed excitation mechanism for 
    % the zonally elongated phase lines in the horizontal cross sections of Figure \ref{fig:RF25_era5_horizonal}. 
    % long vertical wave lengths in vertical cross section and elongated phase line in horizontal
    % Address vertical cross section in introduction
\end{tcolorbox}
It refers to the zonally elongated phase lines in the horizontal cross sections of Figure \ref{fig:RF25_era5_horizonal} observed by \textcite[]{dornbrack_stratospheric_2022}. At first, Section \ref{sec:3D_noshear} demonstrated that the orientation of the tropopause depression alone can not produce a comparable GW pattern to Figure \ref{fig:RF25_era5_horizonal} and the wave vector remains almost parallel to the zonal flow. By contrast, the meridional wind shear introduced in Section \ref{sec:3D_shear} can significantly alter the orientation of the wave vector and we conclude that a very strong PNJ located in the south of a tropopause depression leads to zonally elongated phase lines in the upper stratosphere that mimic the appearance of GWs in Figure \ref{fig:RF25_era5_horizonal}. So NOGWs above propagating depressions could explain the GW structures in the upper stratosphere above the Southern Ocean. \\
The zonal width of the tropopause depression only influences momentum and enery fluxes in the context of changing scales, but it does not seem to significantly alter the orientation of the wave vector. On the contrary, a horizontal rotation of the depression has a counterintuitiv effect and leads to a larger angle $\phi$ between the phase lines and the x-axis. An increasing width in the direction of the incoming flow dampens the development of $l$, but it can not fully explain this phenomenon.   

It could be educational to investigate additional variations of the PNJ's meridional position on the angle $\phi$. Potentially, a PNJ closer to the tropopause depression (at for example $y=\SI{-750}{\kilo\meter}$) results in a smaller angle $\phi$.

- Add comment on momentum flux calculations via wind or temperature perturbations... maybe include it in a research question or mention in introduction?

%These results provide ad- ditional evidence of a widespread oblique propagation ef- fect, described in an increasing number of studies (Watanabe et al., 2008; Wu and Eckermann, 2008; Preusse et al., 2009; Sato et al., 2009, 2012; Ern et al., 2011; Kalisch et al., 2014; Hindley et al., 2015; Alexander et al., 2016; Ehard et al., 2017). 

% Motivated by the work of   Figure \ref{fig:RF25_era5_vertical}
% Phase lines in vertical cross section triggered the analogy to MWs


%%%%%%%%%%%%%%%% COMMENT (PRELIMINARY RESULTS) %%%%%%%%%%%%%%%%%%%%%

\begin{comment}



\subsection{Linear regimes}
\label{sec:linearRegimes}


take description of waves from gill and queney!! 

hydrostatic case waves above mountain with wavelength....

non-hydrostatic lee waves 




\begin{table*}[h]
\centering
\caption{Parameters for the numerical simulations: mountain width L, spatial increments $\Delta$x and $\Delta$z in the horizontal and vertical directions, time step $\Delta$t, thickness $\delta$x$_{ab}$ and time scale $\tau_x$ of the horizontal and altitude z$_{ab}$ and time scale $\tau_z$ of the vertical absorbers.}

\begin{tabular}{@{}lcccccccc@{}}
\toprule
Wave regime & L/km & $\Delta$x/m & $\Delta$z/m & $\Delta$t/s & $\delta$x$_{ab}$/km & $\tau_x$/s  & z$_ab$/km & $\tau_z$/s \\ \midrule[1pt]

Non-hydrostatic & 1 & 100 & 100 & 5 & 24 & 1800 & 38 & 900 \\
Hydrostatic & 10 & 1000 & 100 & 5 & 240 & 300 & 38 & 900 \\
Inertia-Gravity & 100 & 5000 & 100 & 60 & 1200 & 300 & 38 & 3600 \\

\bottomrule
\end{tabular}
\label{tab:linearRegimes}
\end{table*}

\begin{itemize}
    \item h$_m$ = 100 m
    \item U = 10 m s$^{-1}$
    \item N = 0.01 $s^{-1}$ (Brunt–Väisälä frequency)
    \item f = 1.03$\cdot 10^{-4}$ = 1.457$\cdot 10^{-4}$ $\cdot \sin{45°} \approx$ 0.01 N (compare with figure from Gill)
\end{itemize}


\subsection{Comparison of surface shapes} %Q Witch of Agnesi vs. Cos^4

The background flow, or more precisely the wind speed u and stability N (Brunt-Vaisalla frequency) of the atmosphere, define the vertical wavelength of hydrostatic waves that develop for flows over a mountain ridge (Equation \ref{equ:lambdaz}).

\begin{equation}
    N^2 = g \cdot st
    \label{equ:N}
\end{equation}

\begin{equation}
    \lambda_z = \frac{2\pi U}{N}
    \label{equ:lambdaz}
\end{equation}


\subsection{Perturbations in a quiescent fluid}

% Simplifications:

hydrostatic assumptions
geostrophic assumption

only valid as long as perturbations are small (linearization)

Bousinessq approximation

taylor Goldstein equation

\cite{NappoOrLin2007}

(constant buoyancy frequency N of 0.01 $\frac{1}{\textrm{s}}$)

Group velocity parallel to constant phase lines... -> show energy propagation and phase lines

phase velocity perpendicular to group vel

wave number vector points in direction of phase velocity

energy upward and right, phase downward and right


\begin{equation}
    c_x = \frac{\omega}{k} = u_0 \pm \frac{N}{\sqrt{k^2+m^2}}
    \label{equ:cx}
\end{equation}

\begin{equation}
    c_z = \frac{\omega}{z} = u_0 \frac{k}{m} \pm \frac{k N}{m\sqrt{k^2+m^2}}
    \label{equ:cz}
\end{equation}

\begin{equation}
    c_{gx} = u_0 \pm \frac{N m^2}{(k^2+m^2)^{\frac{3}{2}}}
    \label{equ:cgx}
\end{equation}

\begin{equation}
    c_{gz} = u_0 \pm \frac{N k m}{(k^2+m^2)^{\frac{3}{2}}}
    \label{equ:cgz}
\end{equation}


% \include{timeSchedule}





\begin{enumerate}[label=\Alph*]
    \item Verification
    \begin{enumerate}[label=\roman*] % (\arabic*)
        \item Elementary flow regimes
        \item 2D moving lower topography
    \end{enumerate}
    \item Tropopause represents lower boundary of simulation
    \begin{enumerate}[label=\roman*] % (\arabic*)
        \item 2D simulation of moving tropopause depression
        \item 3D simulation of moving tropopause depression without polar night jet
        \item 3D simulation of moving tropopause depression with polar night jet
    \end{enumerate}
    \item Full simulation including troposphere part of simulation
\end{enumerate}


\end{comment}