\chapter*{Preface}
\addcontentsline{toc}{chapter}{Preface}
\thispagestyle{plain}
%
% \LaTeXe{}
%
% "Sometimes, you just have to go with the waves." It kind of fits to my process of deciding on a thesis topic. I remember driving back home from the Atlantic, when reading an e-mail from Bernd about possible subjects. One of them even included the investigation of wave breaking
%
\textit{"You can't stop the waves, but you can learn to surf."}
%
%I like this quote by American medicine professor Jon Kabat-Zinn. When surfing, you can decide which wave you want to catch and ride in the ocean and in life. 

I like how this quote by American medicine professor Jon Kabat-Zinn applies to surfing but also life in general. When you learn to "surf", you can decide which wave you want to catch and ride in the ocean and in many fields of life. 

I remember driving home from a surf trip along the Atlantic coast when reading an e-mail from Bernd about possible master thesis topics within DLR's Middle Atmosphere group in the fall of 2020. All subjects were related to gravity waves. One even included the investigation of wave breaking. Waves in the atmosphere can break? Break like ocean waves I tried to surf the past three weeks? I barely knew anything about gravity waves at that point in time and dreamed more about my next surf session than anything else when thinking about it. Consequently, it was a simple decision. This seemed like a wave I wanted to catch.

About two months later, Alexander introduced gravity waves in his mountain meteorology lecture, and I became increasingly familiar with the topic. Time passed, and in the subsequent spring, I sat at Andreas' dining table with a cup of tea, and he introduced me to his work on gravity waves above propagating tropopause depressions. 

As far as I can tell today, I caught a fantastic wave with many exciting and smooth sections. Other sections were challenging, and partially, other waves, like working for the avalanche warning service, interfered and everything took much longer than planned. But finally, it is finished, and I am deeply grateful for the patience and support of my supervisors, Andreas and Alexander. Let's see where this wave keeps taking me.


% From then on, it was more
% his waves  possible investigations for the master's thesis.

% This \LaTeXe{} template is based on my own dissertation. It has been extensively
% modified within the framework of the course \emph{Introduction to Scientific
% Working} which I taught at the University of Innsbruck in the winter semester
% 2009/2010 for students of the \emph{Atmospheric Sciences Master Program}. It
% does \emph{not} serve as the official thesis template of the Institute of
% Meteorology and Geophysics (IMGI), however, it follows some reasonable rules,
% such as the reference and citation guidelines of the American Meteorological
% Society. Before using this template at IMGI, make yourself familiar with the
% format guidelines of the Faculty of Geo- and Atmospheric Sciences and
% the personal preferences of your advisor. My \LaTeX{} knowledge is based on the
% guides of \citet{oeti08Aag} and \citet{kopk99Aag}. Many ideas on the content and
% structure of a science thesis are taken from the book of \citet{russ06Aag}. This
% template is distributed in the hope that it will be useful, but \emph{without
% any warranty}. I am looking forward to receiving comments for
% improvements.\footnote{\texttt{alexander.gohm [at] uibk.ac.at}}

% "You can't stop the waves, but you can learn to surf."
% Jon Kabat-Zinn

% "Home is where the waves are."
% Unknown Author

% "Sometimes, you just have to go with the waves."
% Unknown Author

% "Feelings are much like waves: we can't stop them from coming, but we can choose which one to surf."
% Jonatan Mårtensson

\begin{flushright}
\textit{M. B.}

\textit{Innsbruck, March 2023} 
\end{flushright}
