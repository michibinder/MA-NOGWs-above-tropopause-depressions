\chapter{Methods and data}
\label{sec:methods}
The following subsections provide a sufficient overview on the numerical model and analysis tools used for the presented work to allow a conclusive interpretation of the results thereafter.

\section{Spectral filtering}
\label{sec:spectral_filter}
% FFT

% and from Vera

- edge padding vs. symmetric padding. Pads with the reflection of the vector mirrored along the edge of the array.

- edge padding might overestimate changes in fluxes close to the surface.

- reflection / extension based on cutoff wavelength results in most realisitc estimation of fluxes 

- compare Gaussian filter vs. butterworth filter

- refer to filtering described in appendix of \textcite[]{kruse_gravity_2015}

- describe butterworth filter for CORAL data

\section{Wavelet Analysis}
\label{sec:wavelet}
The wavelet analysis described by \textcite{torrence_practical_1998} is used for deriving horizontal and vertical wavelengths of GWs in the idealized numerical simulations. 

% \begin{equation}
%     \mathcal{F}_x =  \rho_0 \int_{-\infty}^{\infty} (u'\omega' + f v' \eta') dx + f \int_{-\infty}^{\infty} \rho' v' \eta' dx
%     \label{equ:morlet}
% \end{equation}


- The normalization with the variance of the data $\sigma^2$ gives a measure of the power relative to white noise. 

- also describe red noise and Significance level

- 

\begin{figure*}[t]
    \centering
    \includegraphics[width=0.8\textwidth]{figures_methods/waveletAna_power_spectrum.png}
    \caption{}
    \label{fig:wavelet_example}
\end{figure*}


% Mithilfe der Wavelet-Transformation können aus einer Datenreihe nicht nur die auftre- tenden Frequenzen extrahiert werden, sondern auch Informationen darüber, in welchen Abschnitten der Datenreihe welche Frequenzen dominant sind. Als Basisfunktionen wer- den dabei räumlich lokalisierte Wellen, sogenannte Wavelets, verwendet.

% Folgenden wird die Wavelet-Analyse anhand einer Datenreihe f(z) erläutert, die ent- lang einer räumlichen Achse z variiert. Dabei wird auf die Beschreibung in Torrence & Compo (1998) zurück gegriffen. Die Autoren stellen auf der Website http://atoc.colorado.edu/research/wavelets/ Software zur Anwendung der Wavelet-Ana- lyse zur Verfügung, die in dieser Arbeit verwendet wurde.
% Hier wurde das Morlet-Wavelet ψ0(η) benutzt, das von einem dimensionslosen Ortspa- rameter η abhängt und als
% ψ0(η) = π−1/4 ei ω0η e−η2/2 (3.39)
% definiert ist. Real- und Imaginärteil von ψ0 sind in Abbildung 3.6a dargestellt. Wird f(z) als eine Datenreihe fj diskretisiert, die bei konstanter Intervallgröße ∆z auf einem Gitter mit Index j = 0,...,N−1 definiert ist, kann daraus die kontinuierliche Wavelet- Transformation Wn(s) ermittelt werden. Diese ist definiert als Faltung von fj mit einer skalierten und verschobenen Version der Wavelet-Funktion:
% N−1 Wj(s) = �� fj′ ψ∗
% ��(j′ −j)∆z�� s
% (3.40) die komplex konjugierte normierte Wavelet-Funktion ψ0.
%  Hierbei ist ψ
% ∗
% ���� ∆z ��1/2 ��∗ = s ψ0
% j′=0
%  Indem die Wavelet-Skala s variiert und ψ∗ entlang des Ortsindex j verschoben wird, kann ein Bild rekonstruiert werden, das sowohl die Amplitude einzelner Merkmale des Signals zeigt als auch die Variation der Amplitude mit dem Ort.
% Für jede Skala s muss Gleichung (3.40) N-mal angewandt werden, damit die kontinu- ierliche Wavelet-Transformation approximiert wird. Diese Berechnung erfolgt deutlich schneller im Fourier-Raum, wo die Wavelet-Transformation gleichzeitig für alle N durch- geführt werden kann. Für die Skalen s empfiehlt sich eine Wahl von M Skalen, die als Vielfache von 2 ausgedrückt werden:
% sm = s0 2m∆m mit m = 0,1,...,M und M = ∆m−1 log2(N ∆m/s0)
% Die kleinste Skala s0 sollte dabei so gewählt werden, dass die entsprechende Fourier-
% Periode etwa 2∆z beträgt. Aus der komplexen Wavelet-Transformierten Wj(s) kann
% das reelle Wavelet-Leistungsspektrum |Wj(s)| berechnet werden. Bei der Interpretation
% dieses Spektrums muss beachtet werden, dass bei der Fourier-Transformation eine An-
% nahme bezüglich zyklischer Daten gemacht wird, die nicht unbedingt erfüllt ist. An den
% Rändern des Datensatzes können deshalb Fehler auftreten. Der Einflusskegel (engl.: cone
% of influence) gibt an, in welchen Bereichen des Spektrums Randeffekte wichtig werden.
% Er ist definiert als e-Abklingzeit τs der Wavelet-Leistung zu jeder Skala s und für das √
% Morlet-Wavelet gilt τs = 2 s.
% Im rechten Bildteil von Abbildung 3.6c ist ein Wavelet-Leistungsspektrum dargestellt, in dem auch der Einflusskegel als weiße Linie markiert ist. Die schwarze Kontur gibt ein Konfidenzniveau von 95 % an, bezogen auf ein Spektrum roten Rauschens mit lag-

% 1-Koeffizient 0.72 (siehe Torrence & Compo, 1998). Es zeigt die Wavelet-Analyse ei- nes Profils des Vertikalwinds (mittlerer Bildteil von Abbildung 3.6c) aus einer EU- LAG-Simulation. Der Vertikalschnitt durch das Windfeld ist in Abbildung 3.6b als rot-gestrichelte Linie dargestellt. Im Wavelet-Leistungsspektrum, das mit dem Morlet- Wavelet (Abbildung 3.6a) erstellt wurde, können einzelne Bereiche als dominante Signale ausgemacht werden. Im unteren Bereich bis zu einer Höhe von z = 10 km ist eine ver- tikale Wellenlänge λz zwischen etwa 2 000 m und 3 000 m verstärkt vorhanden, während in größeren Höhen kleinere Wellenlängen von unter 1000m das Spektrum bestimmen. Dieser Fall ist in Abschnitt 4.2.1 ausführlich besprochen.


% include figure from wavelet analysis for one line in 2D data, show line in T' plot..

% Figure: wavelet + cut + power spectrum 


\section{CORAL temperature measurements} 
\label{sec:coral}

\textcite{kaifler_compact_2021} provide a more extensive summary on CORAL's setup and observation,


\section{(ERA5 reanalysis data)}

In Chapter \ref{cha:lidar} we use the ERA5 reanalysis introduced by \textcite[]{hersbach_era5_2020} for two case studies on GW observations by the ground-based lidar CORAL in Río Grande at the southern tip of south 
% Within the Copernicus Climate Change Service (C3S), ECMWF is producing the ERA5 reanalysis which, once completed, will embody a detailed record of the global atmosphere, land surface and ocean waves from 1950 onwards. This new reanalysis replaces the ERA-Interim reanalysis (spanning 1979 onwards) which was started in 2006. ERA5 is based on the Integrated Forecasting System (IFS) Cy41r2 which was operational in 2016. ERA5 thus benefits from a decade of developments in model physics, core dynamics and data assimilation. In addition to a significantly enhanced horizontal resolution of 31 km

Copernicus Climate Change Service (C3S) Climate Data Store (CDS)

cite Hersbach paper and ML dataset for temperature perturbations 

winds on 850hPa

and height of dynamical tropopause (2 PVU level)

and d
- maybe also comment on observational filter of satellite measurments from with figure from Hindley