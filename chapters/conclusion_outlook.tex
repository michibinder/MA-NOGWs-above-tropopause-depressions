\chapter{Conclusion}

% Answer research questions from input!!

2D
- A reference wind speed (wind in total column - spped of transient lower boundary) can be used to generate similar wave pattern in stationary reference frame

- conclusion from 2D cases -> transfer of stationary mountain wave analysis
- valve layer! do dissipation analysis with reference wind!!

3D
- Background conditions in stratosphere can result in wave pattern observed in the case study by \cite{dornbrack_stratospheric_2022} during RF 25 of the DEEPWAVE campaign.
- Spectrum fits in terms of observed wavelenghts

- meteorlogical parameters can be observed all around jet regions 
 - what are expected wavelengths from geostrophic adjustment?

Wavelength discussion:
AIRS data:
Sensitivity is close to 100 percent for waves with wavelengths between 35 and 45 km in the vertical and less than around 500 km in the horizontal, and around 50 percent for wavelengths greater than 17 km in the vertical and less than 1000 km in the horizontal. The majority of our measured wavelengths in the results of this study fall within this 50 percent sensitivity region (as we would expect),

Regarding the GW belt we have to consider that AIRS is mainly sensitive to vertical wavelengths larger than 17km. These wavelengths almost exclusively occure in the vicinity of the northern or southern PNJ due to the refraction of MWs or NOGWs into the jet. The jet favors the propagation of GWs towards higher altitudes (stratosphere/mesosphere), but on the other side the transformation of the wave properties leads to a significant influence of the istrument's observational filter.
\textcite[]{hindley_gravity_2019} show that 

Especially tropopause folds with smaller widths reproduce a GW structures similar to patterns in the ERA5 data (horizontal wavelengths fit and tilted phase lines due to meridional shear resulting in elongated structures in horizontal cross section) \textcite[]{dornbrack_stratospheric_2022}
eventually reference figures from introduction.


Same conclusion as many studies that oblique propagation of GW is significant and has to be improved in GCMs. Parameterisations like recent from Eichinger, 


% However, the belt of increased flux over the Southern Ocean shown in Ern et al. (2018) appears to be compara- tively more pronounced during June to August in their study than we observe here in AIRS measurements. The observa- tional filter of limb-sounding instruments means that they are more sensitive to gravity waves with relatively short ver- tical wavelengths (∼ 3–15 km) and relatively long horizon- tal wavelengths (∼ 500–5000 km). This suggests that a sig- nificant part of the oceanic section of the belt of enhanced gravity wave activity at 60◦ S is made up of long-horizontal wavelength waves, to which AIRS is less sensitive. \cite[]{hindley_gravity_2019}

% Critical assumption 
% Ina (ECMWF 1km simulation) suggests that gravity wave belt difference is associated with <100km waves
% What wavelengths are suggested based on AIRS data


%%%%%%%%%%%%% \chapter{Outlook} %%%%%%%%%%%%%%

- full 3D simulations (cite zhang.. 2005 together with Alexander/Durran...) extend those simulations to upper stratosphere.

- further LIDAR analysis

