\chapter{The numerical model EULAG}
\label{sec:EULAG}
The nonlinear numerical simulations are conducted with the EUlerian/semi- LAGrangian fluid solver (EULAG). EULAG was written by Piotr Smolarkiewicz, but many others contributed to the code. The research code is comprised in one file combining two programming languages, shell script and FORTRAN (mostly FORTRAN 77), to create a very fast executable program. A comprehensive description of the algorithm is given in \textcite[]{smolarkiewicz_forward--time_1997} and \textcite[]{smolarkiewicz_mpdata_1998}. \\
The underlying setup of EULAG for this work is described in the first section of this chapter. It follows a more detailed description of relevant atmospheric background conditions in section \ref{sec:ambient-profiles} and the implementation of a transient lower boundary for the idealized simulations is explained in section \ref{sec:trans-boundary}. Section \ref{sec:linear-MWs} completes this chapter with a comparison of non-linear simulations with EULAG to analytic results of simplified MW scenarios. It serves as an additional validation of EULAG before using it for the investigation of NOGWs above propagating tropopause folds.

\section{General setup}
\label{sec:eulag-setup}
The model is set up solving the soundproof anelastic set of equations (\cite{lipps_scale_1982}) consisting of the three components of the momentum equation (\ref{equ:momEqu}), the thermodynamic equation (\ref{equ:potTemp}) for the potential temperature perturbation $\theta$'=$\theta-\theta_e$ and the mass continuity equation (\ref{equ:continuityEqu}):
%
\begin{equation}
\begin{aligned}
    \frac{d \Vec{v}}{dt} = -G \Vec{\nabla} (\frac{p'}{\bar{\rho}}) +  \Vec{g} \frac{\theta'}{\bar{\theta}} - 2 \Vec{\Omega} \times (\Vec{v}-\Vec{v_e}) \\
    - \Tilde{\alpha} (\Vec{v}-\Vec{v_e}) \equiv R^{v},
    \label{equ:momEqu}
\end{aligned}
\end{equation}
%
\begin{equation}
    \frac{d \theta'}{dt} = -\Vec{v} \cdot \Vec{\nabla} \theta_e - \Tilde{\beta} (\theta-\theta_e) \equiv R^{\theta},
    \label{equ:potTemp}
\end{equation}
%
\begin{equation}
    \Vec{\nabla} \cdot (\bar{\rho} \Vec{v}) = 0.
    \label{equ:continuityEqu}
\end{equation}
%
Here, $\frac{d}{dt}$, $\Vec{\nabla}$ and $\Vec{\nabla} \cdot$ represent the total derivative, the gradient and the divergence respectively. $p'$ is the pressure perturbation with respect to the environmental state, g the gravitational acceleration and $\Vec{\Omega}$ the angular velocity of the Earth. The matrix G represents geometric terms, which result from the general, time-dependent coordinate transformation and the symbol $R^{\Psi}$ stands for the right-hand side of the corresponding equations for the variables $\Psi = (u,v,w,\theta')$.
% with $H_{/rho} = 7314m$. 
refer to the hydrostatic reference state around a constant stability profile as introduced by \textcite{bacmeister_breakdown_1989} with stability $\frac{N^2}{g}$, a density scale height $H_{\rho}$ that corresponds to a deep atmosphere and $\rho_0$, $\theta_0$ set to appropriate reference constants. With (\ref{equ:thetaScale}) and (\ref{equ:densityScale}) being the basic state of equations (\ref{equ:momEqu}-\ref{equ:continuityEqu}), a more general environmental state, which reflects the initial and boundary conditions, enters the equations via the variables with subscript e. In that sense, $\alpha(\Vec{v}-\Vec{v_e})$ and $\beta(\theta-\theta_e)$ represent relaxation terms, which enable the radiation of wave energy across the model boundaries and force the solutions at the model boundaries to the prescribed environmental profiles. These ambient states like $u_e$, $\rho_e$ or $\theta_e$ can be time-dependent to replicate transient flow conditions, but are stationary within the scope of this thesis. On the other hand, transient boundaries like a propagating tropopause fold almost demand for time-dependent terrain-following vertical coordinates as introduced by \textcite{wedi_extending_2003}, so this option may be considered at a later stage. Furthermore, EULAG is noteworthy for its robust elliptic solver (\cite{smolarkiewicz_forward--time_1993}) and generalized coordinate formulation enabling grid adaptivity technology (\cite{prusa_eulag_2008}, \cite{kuhnlein_modelling_2012}).

As the name suggests, EULAG is capable of solving the equations of motion (\ref{equ:momEqu}-\ref{equ:continuityEqu}) in an Eulerian (flux form) or in a semi-Lagrangian (advective form) mode (\cite{smolarkiewicz_forward--time_1997}). For the numerical approximation it utilizes a non-oscillatory forward-in-time (NFT) approach compactly formulated as
%
\begin{equation}
\begin{aligned}
    \Psi^{n+1} = LE(\Tilde{\Psi}, V^{n+1}, G^n, G^{n+1}) \\
    + \frac{1}{2} \Delta t R^{\Psi} |^{n+1}
    \label{equ:NFTscheme}
\end{aligned}
\end{equation}
%
with $LE$ representing the corresponding semi-Lagrangian/Eulerian transport operator. The NFT scheme belongs to the class of second-order-accurate two-time-level algorithms that are build on nonlinear advection techniques (\cite{prusa_eulag_2008}). These schemes have the property to suppress and control numerical oscillations that are often found in higher order linear schemes. As a result, transporting the auxiliary field $\Tilde{\Psi} = \Psi^n + \frac{1}{2} \Delta t R^{\Psi}|^n$ instead of the specific variable $\Psi$, results from a thorough truncation error analysis and ensures second order accuracy. (\cite{smolarkiewicz_forward--time_1997}).

Within the scope of this Master's thesis all simulations utilize the Eulerian option by applying the multidimensional positive definite advection transport algorithm MPDATA (\cite{smolarkiewicz_mpdata_1998} and  and \cite{smolarkiewicz_multidimensional_2006}).

EULAG has a proven itself as a reliable tool for simulating thermo-fluid flows across the wide range from turbulent to global scales (\cite{prusa_all-scale_2003}) and in a variety of of physical scenarios like e.g. turbulence, GW dynamics, flows past complex/moving boundaries, micrometeorology or cloud microphysics (\cite{prusa_eulag_2008}). A comparison between different well-established numerical models (including EULAG) and their capability to model flow over steep terrain, which is relevant for the investigations within this thesis, appears in \textcite{doyle_intercomparison_2011}.

%%%%% NOTES %%%%%
% Prusa 2008 also has good description of perturbation form and points out importance of correct environmental/initial state!! 

% \cite{smolarkiewicz_multidimensional_2006}

% \begin{equation}
%    \frac{\partial G \rho \Psi}{\partial t} + \nabla \cdot G \rho \Vec{v} = G \rho R
%    \label{equ:statMeshAdapt}
% \end{equation}


% he elliptic pressure equation is solved via a preconditioned non-symmetric Krylov sub- space solver (Smolarkiewicz and Margolin; 1994; 1997, Skamarock et al., 1997).

% imlicit /explicit elliptic pressure solver / krylov sub space solver

% There, governing equations are formulated in general- ized time-dependent curvi-linear coordinates to enable mesh adaptivity (Prusa and Smolarkiewicz 2003; Wedi and Smolarkiewicz 2004; Kühnlein et al. 2012; Smolarkiewicz and Charbonneau 2013) and continuous mappings of the Earth’s topography by using terrain-following coordinates (Gal-Chen and Somerville 1975; Clark 1977)


% anelastic -> elastic energy is not allowed -> sound waves are filtered since they are based on pressure difference rather then temperature


\section{Ambient profiles of idealized simulations}
\label{sec:ambient-profiles}
% mark variables to be defined in table!! like in isentropic paper lightcyan
profile of potential temperature (definition of N) sets up background profiles


choice of kappa is not relevant for simulations with using the soundproof anelastic equations similar to \textcite[]{lipps_scale_1982}. The correct value for diatomic gases is $\kappa=\frac{2}{7}$. For simulations of the stratosphere with N=0.02 This leads to a reasonable temperature of 239K, but for N=0.01, a reasonable value for the troposphere, ambient profiles following \textcite{bacmeister_breakdown_1989} would result in T=900k. Therefore, T = 290K... with K=

\begin{figure*}[tbp]
    \centering
    \includegraphics[]{figures_model/bac-schoeber-ambient-profiles.png}
    \caption{Hssss}
    \label{fig:ambient_profs}
\end{figure*} 

\begin{figure*}[tbp]
    \centering
    \includegraphics[]{figures_model/eulag-wind-profiles.png}
    \caption{Vertical profiles of the background wind for EULAG simulations with no meridional shear (a) and with meridional shear (b). The black line in (a) is a superposition of the tropopause jet (purple) and the PNJ (yellow). Both individual wind profiles are based on equation... The 2-D wind profiles in (b) are obtained by combining the vertical profiles from (a) with a meridional distribution (red curve in (b) for the PNJ at 60°S) based on equation..., too.}
    \label{fig:wind_profs}
\end{figure*} 


EULAG provides multiple options to define background states of the atmosphere. For the presented simulations vertical ambient profiles define an isothermal atmosphere with constant stability as described by \textcite{bacmeister_breakdown_1989}. These exponential profiles of potential temperature and density avoid physical restrictions towards higher altitudes and are thus well suited for the investigation of deep gravity wave propagation. Defining a potential temperature and density scale height $H_{\Theta}$ and $H_{\rho}$ leads to


\begin{equation}
    \bar{\theta}(z) = \theta_0 \textrm{ exp}(-\frac{N^2}{g} z) 
    \label{equ:thetaScale}
\end{equation}
%
and the anelastic density 
%
\begin{equation}
    \bar{\rho}(z) = \rho_0 \textrm{ exp}(-\frac{z}{H_{\rho}})
    \label{equ:densityScale}
\end{equation}
% with $H_{/rho} = 7314m$. 
refer to the hydrostatic reference state around a constant stability profile as introduced by \textcite{bacmeister_breakdown_1989} with stability $\frac{N^2}{g}$, a density scale height $H_{\rho}$ that corresponds to a deep atmosphere and $\rho_0$, $\theta_0$ set to appropriate reference constants.

\begin{equation}
\begin{aligned}
    p_0(z) &= p_{00} e^{-\frac{z}{H_{\rho}}} \quad \textrm{with} \quad H_{\rho} = \frac{R_d}{c_p} H_{\Theta} = \frac{R_d T_{00}}{g} \\
    \rho_0(z) &= \frac{p_0(z)}{R_d T_{00}} = \rho_{00} e^{-\frac{z}{H_{\rho}}} \\
    T_0(z) &= \frac{p_{00}}{R_d \rho_{00}} \\
    \Theta_0(z) &= T_{00} \frac{p_{00}}{p}^{\kappa} \\
    \Theta_0(z) &= T_{00} e^{\frac{z}{H_{\Theta}}} \quad \textrm{with} \quad H_{\Theta} = \frac{g}{N^2} \\
    \label{equ:ambient-Profiles}
\end{aligned}
\end{equation}
% based on hydrostatic approximation dp/dz = -rho*g -> pressure with density scale height

with the Brunt-Vaisala frequency $N$, the specific gas constant $R_d$ and the specific heat capacity at constant pressure $c_p$. $p_0$, $\rho_0$ and $T_0$ represent a reference state of the atmosphere.

\begin{table*}[ht]
\centering
\caption{Parameters related to the ambient reference state and vertical profiles following \textcite[]{bacmeister_breakdown_1989}. Values for the troposphere are used for the model validation in section \ref{sec:linear-MWs}. Values for the stratosphere are used for the main simulations of this thesis described in chapters \ref{sec:resultsQ3D} and \ref{sec:results3D}. Colored cells refer to parameters that need to be defined.}

%%%% include trop for diatomic gas - then explain difference
%%% do scaled values still make sense
\begin{tabular}{@{}cccc@{}}
\toprule
 & Unit & Troposphere & Stratosphere \\ \midrule[1pt]

$R$ & J kg$^{-1}$ K$^{-1}$ &   \cellcolor{LightCyan} 287.04 &   \cellcolor{LightCyan} 287.04 \\
$g$ & m s$^{-2}$ & \cellcolor{LightCyan} 9.80616 & \cellcolor{LightCyan} 9.80616 \\
$N$ & s$^{-1}$ & \cellcolor{LightCyan} 0.01 & \cellcolor{LightCyan} 0.02 \\
$H_{\Theta}$ & m & 98061.6 & 24515.4  \\
$H_{\rho}$ & m & 28017.6 (8011.3)  & 7004.4 \\
$\kappa$ & - & $ \frac{2}{7}$ ($\frac{2}{24.4}$) & $ \frac{2}{7}$ \\
$c_p$ & J kg$^{-1}$ K$^{-1}$ & $\frac{7}{2} R$ ($\frac{24.4}{2} R$) & $\frac{7}{2} R$ \\

& & & \\
$T_{00}$ & K & \cellcolor{LightCyan} 957.17 (273.69) & 239.39 \\
$p_{00}$ & Pa &  \cellcolor{LightCyan} $1.01 \cdot 10^5$ & \cellcolor{LightCyan} $0.235 \cdot 10^5$ \\
$\rho_{00}$ & kg m$^{-3}$ & 0.3676 (1.2856) & 0.3454 \\

% kappa = H_rho/H_th 

\bottomrule
\end{tabular}
\label{tab:ambientProfiles}
\end{table*}

Should density scale height fit to reference state or should it rather relate correctly to stability / theta scale height based on two atomic gases? This results in temperature around 900K...
Value of density scale height effects

For stratosphere simulations it s fully consistent!!! 

- full 3D simulations (cite zhang.. 2005 together with Alexander/Durran...) extend those simulations to upper stratosphere.


\section{Transient lower boundary of idealized simulations}
\label{sec:trans-boundary}
\begin{figure*}[tbp]
    \centering
    \includegraphics[width=0.67\textwidth]{figures_model/topo-transient-boundary.png}
    \caption{Shown are different variations of a $(1+cos(x))$ mountain (black solid and dotted lines) and a Witch of Agnesi (red dashed line). The vertical dimension is scaled by the mountain height $h_m$, the horizontal dimension by the mountain half width $a$. It allows a comparison of all shapes independent of $h_m$ and $a$.}
    \label{fig:topo_trans}
\end{figure*} 
The Witch of Agnesi
\begin{equation}
    h_{Agnesi}(x) = h_m \frac{a^2}{x^2+a^2}
    \label{equ:agnesi}
\end{equation}
is an established shape to describe isolated mountains in idealized simulations with the crest height $h_m$ and half width $a$. It was already used by \textcite{queney_problem_1948} to derive linear analytic solutions to simplified MW scenarios. EULAG simulations with the Witch of Agnesi are compared to Queney's solution in the next section to validate the model. \\
However, the topography following equation \ref{equ:agnesi} is non-zero for the whole domain of a discrete simulation. We recap that it is the goal of this work to conduct simulations with a transient lower boundary that mimics a propagating tropopause fold. In these simulations a Witch of Agnesi would lead to a constantly changing surface at the domain's horizontal borders and impact the boundary conditions. This makes it a less appropriate shape for a moving topography or transient boundary in simulations. \\
A suitable alternative is a $1+cos(\frac{\pi}{4a}x)$ function. It drops to 0 for $|\frac{x}{4a}| = 1$, so the sorrounding topography can be set to 0 for $|\frac{x}{4a}| \leq 1$ without sacrificing its continuity and differentiability. This is essential for the implementation of a transient boundary in the model. It exists a variation of the $(1+cos(x))$ function  
\begin{equation}
    h_{cos}(x) = 
    \begin{cases}
        & \frac{h_m}{16} (1+cos(\frac{\pi}{4a}x))^4, |\frac{x}{4a}| < 1 \\
        & 0, |\frac{x}{4a}| \geq 1 \\
      \end{cases}
    \label{equ:cosMtn}
\end{equation}
with a quite comparable shape to the Witch of Agnesi (Equation \ref{equ:agnesi}). Figure \ref{fig:topo_trans} shows how the shapes compare for a similar half-width and crest height and makes evident that Equation \ref{equ:cosMtn} is closest to the Witch of Agnesi. As long as the cosine mountain or depression is not too close to the edges the topography does not interfere with horizontal boundary conditions and all contributions to the surface pressure drag are now confined to a small neighbourhood around the peak. \textcite[]{epifanio_three-dimensional_2001} and \textcite{metz_are_2021} already used Equation \ref{equ:cosMtn} for prescribing idealized topographies and \textcite{metz_are_2021} state that the linear pressure drag across the cosine mountain is a factor of 1.3 higher compared to a corresponding Witch of Agnesi for a constant $N$ and $U$ environment.

Growing mountain/valley used to circumvent potential flow initialization.

% Implementation of surface boundary in EULAG (at least show its derivative with x_0,...)
% show how wave propagate -> observe energy propagation / phase lines... simulations without Coriolis in 2D

\section{Non-linear simulations of MW regimes with linear analytic solutions}
\label{sec:linear-MWs}
% for different MW regimes with non-linear numerical simulation}

In his master's thesis Herwig Grogger sho
Grogger also showed EULAG simulations for transient lower boundary, more specifically growing amplitude / rising topography 
Refer to work of Grogger that investigated deep propagating GWs, individual spectral forcings, linear analysis vs numerical results...

% with a constant background flow and stratification

% use small amplitude of 100m to avoid non-linear effect
% Show analysis plots for one simulation and then result of drag for all


% analysis of MFx, EFx, ... profiles for MW regime... thoughts on effect of CORIOLIS!!! MFx not constant...
% Witch of Agnesi mountain - no cos^4
One established approach to validate a numerical model is the comparison of non-linear numerical simulations to results from linear theory. For the investigation of GWs and corresponding processes drag values and momentum flux distributions are suitable.

% - \textcite{blumen} (Blumen 1965a, Fig. I)

- \textcite{bretherton_momentum_1969}

- \textcite{smith_influence_1979} looked circular bell-shaped mountain? only looks at 2D flow of stratified rotating fluid! Flow is independent of y coordinate! 

- Smith assumes symmetric mountain and only looks at one dimension in spectral space?? assumes geostrophic flow? uses gaussian quadrature or bessel functions...

- dimensionless drag is specified differently for Smith and miranda due to additional dimension

- \textcite{miranda_non-linear_1992} obtained the wave drag in a more general non-hydrostatic rotating case and derived an analytic solution for the hydrostatic rotating limit circular bell-shaped mountain. 

- integrate over two dimensions and assume a cross-section with V=0 to calculate drag. Keeps 3D problem first -> equations different to Smith, ...?

- W.K.B


\begin{table*}[ht]
\centering
\caption{Parameters for numerical simulations of different MW regimes: mountain width L, spatial increments $\Delta$x and $\Delta$z in the horizontal and vertical directions, time step $\Delta$t, thickness $\delta$x$_{ab}$ and time scale $\tau_x$ of the horizontal and altitude z$_{ab}$ and time scale $\tau_z$ of the vertical absorbers.}

\begin{tabular}{@{}cccccccc@{}}
\toprule
L/km & $\Delta$x/m & $\Delta$z/m & $\Delta$t/s & $\delta$x$_{ab}$/km & $\tau_x$/s  & z$_{ab}$/km & $\tau_z$/s \\ \midrule[1pt]

1   & 50  & 50 & 5   & 15  & 300  & 24 & 300   \\
2   & 100  & 50 & 5   & 30  & 600  & 24 & 450   \\
5   & 250  & 50 & 10  & 75  & 1200 & 24 & 600  \\
10  & 500 & 50 & 20  & 100 & 1800 & 24 & 900  \\
25  & 1000 & 50 & 40  & 200 & 3600 & 24 & 2400  \\
50  & 1500 & 50 & 60  & 300 & 4500 & 24 & 5400  \\
75  & 2000 & 50 & 60  & 400 & 6000 & 24 & 10500 \\
100 & 2500 & 50 & 75  & 500 & 7500 & 24 & 12000 \\
150 & 3500 & 50 & 180 & 700 & 9000 & 24 & 21000 \\

\bottomrule
\end{tabular}
\label{tab:linearRegimes}
\end{table*}

In that sense, it is the goal to reproduce some fundamental analytic results based on linear theory with nonlinear EULAG simulations. \textcite{queney_problem_1948} was the first to summarize different flow regimes over mountains based on the method of small adiabatic perturbations. His results can be compared to simulations from a visual perspective by looking at the perturbations and fluxes in the atmosphere and from a mathematical one by comparing the wave drag
%
\begin{equation}
    \mathcal{F} = \int_{}^{} p^{'} \frac{\partial h}{\partial x} dx
    \label{equ:waveDrag}
\end{equation}


As described in \textcite{smith_influence_1979}, using the identity
\begin{equation}
    \omega' = U \frac{\partial \eta'}{\partial x}
    % \label{equ:waveDrag}
\end{equation}
together with the linearized streamwise momentum equation for a rotating fluid in the Boussinesq approximation
\begin{equation}
    \rho_0 U \frac{\partial u'}{\partial x} - \rho_0 f v'= -\frac{\partial p'}{\partial x}
    % \label{equ:waveDrag}
\end{equation}
allows a reformulation of the vertical flux of angular momentum (Equation 20 in \textcite[]{bretherton_momentum_1969}) to
\begin{equation}
    \mathcal{F}_x =  \rho_0 \int_{-\infty}^{\infty} (u'\omega' + f v' \eta') dx
    \label{equ:angular_momentum}
\end{equation}

However, Figure ... shows how the GWD from Equation \ref{equ:angular_momentum} deviates from the GWD based on pressure disturbances for MW regimes with larger a mountain width $a$, where the Coriolis effect becomes relevant.
% low frequency waves
Broad 1995 was the first to derive the a linear solution to the GWD by dropping the Boussinesq approximation. An additional term takes the density fluctuations into account resulting in a compressible solution for the streamwise GWD
\begin{equation}
    \mathcal{F}_x =  \rho_0 \int_{-\infty}^{\infty} (u'\omega' + f v' \eta') dx + f \int_{-\infty}^{\infty} \rho' v' \eta' dx
    \label{equ:angular_momentum_compressible}
\end{equation}

The normalized contribution of the compressible part $f \rho' v' \eta'$ was at maximum $\mathcal{O}(10^{-5})$ and can not explain the difference between the GWD from pressure and from the angular momentum.
% \cite[]{teixeira_physics_2014}

Unlike at the surface, where $\eta'=z_{topo}$ it is not possible to calculate Equation \ref{equ:angular_momentum} directly from model data in the free atmosphere. We utilize the approximation
\begin{equation}
    \theta' =  \eta' \frac{\partial \theta}{\partial z}
    \label{equ:thprime}
\end{equation}
to estimate $\eta'$
to its analytic solution in a linear framework (\cite{gill_atmosphere-ocean_1982}). Assuming an impermeable tropopause at the lower boundary of the simulations results in a TD that acts like a flipped mountain (valley) on the stratosphere, so a comparison of our model to Queney's solutions for different mountain wave regimes (non-hydrostatic, hydrostatic, inertia-gravity) provides a reasonable baseline and validation of the model for the planned simulations.

\begin{figure*}[tbp]
    \centering
    \includegraphics[width=0.99\textwidth]{figures_model/GWD-linearTheory-Q3D.png}
    \caption{The force $\mathcal{F}$ is the GWD normalized by ($\rho_0$ N U h$_m^2$)$^{-1}$ for a Witch of Agnesi with half width $L$. Thicker faded lines indicate that the linear solution is based on Bessel functions (\cite[]{blumen_momentum_1965} and \cite[]{smith_influence_1979}). \textcite[]{miranda_non-linear_1992} provide an analytical solution for the low frequency GWs or IGWs. Buoyancy frequency $N$ and flow speed $U$ are uniform and the Coriolis parameter $f=0.01N$. Purple hashes represent the GWD from pressure perturbations, red crosses the Reynolds Stress and green crosses the angular momentum as described in \textcite[]{bretherton_momentum_1969}, \textcite[]{smith_influence_1979} and Broad for different MW simulations.}
    \label{fig:GWD-MWs}
\end{figure*}

% include lessons learned from model validation with vertical resolution for GWD necessary and full 3D simulations required for inertia-gravity waves... though low resolution in cross stream direction good enough.

Focus is on vertical flux of horizontal momentum and energy

For completeness horizontal energy flux and and potential energy are shown. Ep at higher altitudes clearly decreases for increasing mountain widths $a$, but at the surface a minimum exists for the hydrostatic regime with $a \approx \SI{10}{\kilo\meter}$. The horizontal energy flux transitions from being exceptionally increasing to a decreasing $EF_x$ close to the surface for $ 25 < a < \SI{50}{\kilo\meter}$. 

vertical flux of angular momentum is the wave drag in rotating fluid

\begin{figure*}[tbp]
    \centering
    \includegraphics[width=0.99\textwidth]{figures_model/Queney-MWs-zprofiles.pdf}
    \caption{}
    \label{fig:Queney-MWs-zprofiles}
\end{figure*}

% Following \textcite[]{fritts_gravity_2003}... + mid frequency approximation

reynolds stress and the momentum flux in a non-rotational environment or under the mid-frequency approximation.

vertical flux horizontal momentum (flux across a flat horizontal surface )
\begin{equation}
    (\mathrm{MF}_x, \mathrm{MF}_y) = \bar{\rho}  (\overbar{u'w'},\overbar{v'w'})
    \label{equ:mf}
\end{equation}

One aspect


\section*{Lessons learned from non-linear simulations of MW cases}

high vertical resolution important for correct calculation of perturbations at surface

